% CREATED BY DAVID FRISK, 2016
\chapter{Appendix 1}
\section{Lemma 1}
\begin{lemma}
    \begin{enumerate}
    \item $S_1 \ || \ S_2 \equiv S_2\ ||\ S_1$
    \item $(S_1 \ ||\ S_2 )\ || \ S_3 \equiv S_1 \ ||\ (S_2 \ || \ S_3)$
    \item $S_1 \ || \ \Gamma : 0 \equiv S_1$
\end{enumerate}
\end{lemma}
We will proof this lemma by case analysis on $\alpha$. We will have 5 cases with multiple subcases each. To proof a case we need to find a matching transition on both sides which are possible by the semantic rules. Only when all cases and subcases have been covered have we proven the statement.\\
\textbf{1. }The first point is proving commutativity. We can see in the syntax that the cases we have to cover are: $\Pi ! (\Tilde{v})^{ch}$, $\Pi ? (\Tilde{v})^{ch}$, $ \Pi \mathsf{G}(\Tilde{v})^{ch}$, $ \Pi \mathsf{S}(\Tilde{v})^{ch}$ and $ \tau$\\
\textbf{Case 1: } Consider $\alpha = \Pi ! (\Tilde{v})^{ch}$.\\
\indent \textbf{Subcase 1.1: }$S_1 \ || \ S_2 \xrightarrow[]{\alpha} S_1' \ || \ S_2'$. Assume $S_1\not= S_1'$ and $S_2\not= S_2'$, as those cases are covered by subcases 1.2, 1.3 and 1.4. This transition then has two subcases of its own.\\
\indent \indent \textbf{Subcase 1.1.1: } Consider \texttt{Lpar}, we have $S_1 \xrightarrow[]{\Pi ! (\Tilde{v})^{ch}} S_1' $ and $ S_2 \xrightarrow[]{\Pi ? (\Tilde{v})^{ch}} S_2'$. Giving us $S_1 \ || \ S_2 \xrightarrow[]{\Pi ! (\Tilde{v})^{ch}} S_1 ' \ || \ S_2'$. By \texttt{Rpar} on $S_2\ ||\ S_1$, we also get $ S_2 \xrightarrow[]{\Pi ? (\Tilde{v})^{ch}} S_2'$ and $S_1 \xrightarrow[]{\Pi ! (\Tilde{v})^{ch}} S_1' $, resulting in $S_2 \ || \ S_1 \xrightarrow[]{\Pi ! (\Tilde{v})^{ch}} S_2 ' \ || \ S_1'$. Showing $S_1 \ || \ S_2 \equiv S_2\ ||\ S_1$.\\
\indent \indent \textbf{Subcase 1.1.2: } Consider \texttt{Rpar}, we have $S_1 \xrightarrow[]{\Pi ? (\Tilde{v})^{ch}} S_1' $ and $ S_2 \xrightarrow[]{\Pi ! (\Tilde{v})^{ch}} S_2'$. Giving us $S_1 \ || \ S_2 \xrightarrow[]{\Pi ! (\Tilde{v})^{ch}} S_1 ' \ || \ S_2'$. By \texttt{Lpar} on $S_2\ ||\ S_1$, we also get $ S_2 \xrightarrow[]{\Pi ! (\Tilde{v})^{ch}} S_2'$ and $S_1 \xrightarrow[]{\Pi ? (\Tilde{v})^{ch}} S_1' $, resulting in $S_2 \ || \ S_1 \xrightarrow[]{\Pi ! (\Tilde{v})^{ch}} S_2 ' \ || \ S_1'$. Showing $S_1 \ || \ S_2 \equiv S_2\ ||\ S_1$.\\
\indent \textbf{Subcase 1.2: }$S_1 \ || \ S_2 \xrightarrow[]{\alpha} S_1' \ || \ S_2$. In this case only $S_1$ transitions. Looking at the semantics, it does not immediately seem like such a transition exists. However, by \texttt{Nsys} we can discard the $S \xrightarrow[]{\Pi ? (\Tilde{v})^{ch}} S'$ transition making it $\langle \gamma , \mathsf{LS} \rangle : P \xmapsto[]{\widetilde{\Pi ? (\Tilde{v})^{ch}}} \langle \gamma , \mathsf{LS} \rangle : P$.\\ 
\indent \indent \textbf{Subcase 1.2.1: } Consider \texttt{Lpar}, we get $S_1 \xrightarrow[]{\Pi ! (\Tilde{v})^{ch}} S_1' $ and $ S_2 \xrightarrow[]{\Pi ? (\Tilde{v})^{ch}} S_2'$, but for this case by \texttt{Nsys}, we get $S_2=\langle \gamma , \mathsf{LS} \rangle : P$ and $\langle \gamma , \mathsf{LS} \rangle : P  \xmapsto[]{\widetilde{\Pi ? (\Tilde{v})^{ch}}} \langle \gamma , \mathsf{LS} \rangle : P$. This results in $S_1 \ || \ S_2 \xrightarrow[]{\Pi ! (\Tilde{v})^{ch}} S_1' \ || \ S_2$. By \texttt{Rpar} and \texttt{Nsys}, we then get $\langle \gamma , \mathsf{LS} \rangle : P \xmapsto[]{\widetilde{\Pi ? (\Tilde{v})^{ch}}} \langle \gamma , \mathsf{LS} \rangle : P$ for $S_2=\langle \gamma , \mathsf{LS} \rangle : P$ and $S_1 \xrightarrow[]{\Pi ! (\Tilde{v})^{ch}} S_1' $ as well. This results in $S_2 \ || \ S_1 \xrightarrow[]{\Pi ! (\Tilde{v})^{ch}} S_2 \ || \ S_1'$. Showing $S_1 \ || \ S_2 \equiv S_2\ ||\ S_1$.\\
\indent \indent \textbf{Subcase 1.2.2: } Consider \texttt{Rpar}, we would get $S_2$ with the sending transition, but by our IH, we have that $S_2$ does not transition. By \texttt{Nsys}, we see that it is only possible for the receiving, and thus this is not possible.\\
\indent \textbf{Subcase 1.3: }$S_1 \ || \ S_2 \xrightarrow[]{\alpha} S_1 \ || \ S_2'$. In this case we have the opposite of the previous case.\\
\indent \indent \textbf{Subcase 1.3.1: } Consider \texttt{Lpar}, just like in case 1.2.2, this is not possible, because $S_1$ would be the sending system, but we know $S_1$ does not transition.\\
\indent \indent \textbf{Subcase 1.3.2: } Consider \texttt{Rpar}, we would get $S_1 \xrightarrow[]{\Pi ? (\Tilde{v})^{ch}} S_1' $ and $ S_2 \xrightarrow[]{\Pi ! (\Tilde{v})^{ch}} S_2'$, but for this case by \texttt{Nsys}, we get $S_1=\langle \gamma , \mathsf{LS} \rangle : P$ and $\langle \gamma , \mathsf{LS} \rangle : P \xmapsto[]{\widetilde{\Pi ? (\Tilde{v})^{ch}}} \langle \gamma , \mathsf{LS} \rangle : P$. This results in $S_1 \ || \ S_2 \xrightarrow[]{\Pi ! (\Tilde{v})^{ch}} S_1 \ || \ S_2'$. By \texttt{Lpar} and \texttt{Nsys}, we then also get $S_2 \xrightarrow[]{\Pi ! (\Tilde{v})^{ch}} S_2'$ and $\langle \gamma , \mathsf{LS} \rangle : P \xmapsto[]{\widetilde{\Pi ? (\Tilde{v})^{ch}}} \langle \gamma , \mathsf{LS} \rangle : P$ with $S_1=\langle \gamma , \mathsf{LS} \rangle : P$ for $S_2 \ || \ S_1$. This results in $S_2 \ || \ S_1 \xrightarrow[]{\Pi ! (\Tilde{v})^{ch}} S_2 \ || \ S_1'$. Showing $S_1 \ || \ S_2 \equiv S_2\ ||\ S_1$.\\
\indent \textbf{Subcase 1.4:} $S_1 \ || \ S_2 \xrightarrow[]{\alpha} S_1 \ || \ S_2$, this case is a combination of subcases 1.2.2 and 1.3.1. We have explained why for a send transition, at least one system has to transition. This means this transition is not possible.\\
\\
\textbf{Case 2: } Consider $\alpha = \Pi ? (\Tilde{v})^{ch}$.\\
\indent \textbf{Subcase 2.1: }$S_1 \ || \ S_2 \xrightarrow[]{\alpha} S_1' \ || \ S_2'$. Assume $S_1\not= S_1'$ and $S_2\not= S_2'$, as those cases are covered by subcases 1.2, 1.3 and 1.4. Consider \texttt{Prec}, we get $S_1 \xrightarrow[]{\Pi ? (\Tilde{v})^{ch}} S_1' $ and $ S_2 \xrightarrow[]{\Pi ? (\Tilde{v})^{ch}} S_2'$. By \texttt{Prec}, we also get $ S_2 \xrightarrow[]{\Pi ? (\Tilde{v})^{ch}} S_2'$ and $S_1 \xrightarrow[]{\Pi ? (\Tilde{v})^{ch}} S_1' $ for $S_2\ ||\ S_1$, showing $S_1 \ || \ S_2 \equiv S_2\ ||\ S_1$.\\
\indent \textbf{Subcase 2.2: }$S_1 \ || \ S_2 \xrightarrow[]{\alpha} S_1' \ || \ S_2$. In this case only $S_1$ transitions. By \texttt{Nsys} we can discard the $S \xrightarrow[]{\Pi ? (\Tilde{v})^{ch}} S'$ transition making it $\langle \gamma , \mathsf{LS} \rangle : P \xmapsto[]{\widetilde{\Pi ? (\Tilde{v})^{ch}}} \langle \gamma , \mathsf{LS} \rangle : P$. Consider \texttt{Prec}, we get $S_1 \xrightarrow[]{\Pi ? (\Tilde{v})^{ch}} S_1' $ and $ S_2 \xrightarrow[]{\Pi ? (\Tilde{v})^{ch}} S_2'$, by \texttt{Nsys} on $S_2=\langle \gamma , \mathsf{LS} \rangle : P$, we get $\langle \gamma , \mathsf{LS} \rangle : P  \xmapsto[]{\widetilde{\Pi ? (\Tilde{v})^{ch}}} \langle \gamma , \mathsf{LS} \rangle : P$. This results in $S_1 \ || \ S_2 \xrightarrow[]{\Pi ? (\Tilde{v})^{ch}} S_1' \ || \ S_2$. By \texttt{Prec} and \texttt{Nsys} on $S_2$, we also get $S_2 \ || \ S_1 \xrightarrow[]{\Pi ? (\Tilde{v})^{ch}} S_2 \ || \ S_1'$ for $S_2\ ||\ S_1$, showing $S_1 \ || \ S_2 \equiv S_2\ ||\ S_1$.\\ 
\indent \textbf{Subcase 2.3: }$S_1 \ || \ S_2 \xrightarrow[]{\alpha} S_1 \ || \ S_2'$. In this case we have the opposite of the previous case. Consider \texttt{Prec}, we get $S_1 \xrightarrow[]{\Pi ? (\Tilde{v})^{ch}} S_1' $ and $ S_2 \xrightarrow[]{\Pi ? (\Tilde{v})^{ch}} S_2'$, by \texttt{Nsys} on $S_1=\langle \gamma , \mathsf{LS} \rangle : P$, we get $\langle \gamma , \mathsf{LS} \rangle : P  \xmapsto[]{\widetilde{\Pi ? (\Tilde{v})^{ch}}} \langle \gamma , \mathsf{LS} \rangle : P$. This results in $S_1 \ || \ S_2 \xrightarrow[]{\Pi ? (\Tilde{v})^{ch}} S_1 \ || \ S_2'$. By \texttt{Prec} and \texttt{Nsys} on $S_1$, we also get $S_2 \ || \ S_1 \xrightarrow[]{\Pi ? (\Tilde{v})^{ch}} S_2' \ || \ S_1$ for $S_2\ ||\ S_1$, showing $S_1 \ || \ S_2 \equiv S_2\ ||\ S_1$.\\
\indent \textbf{Subcase 2.4:} $S_1 \ || \ S_2 \xrightarrow[]{\alpha} S_1 \ || \ S_2$. This is a combination of subcase 2.2 and 2.3. Consider \texttt{Prec}, we get $S_1 \xrightarrow[]{\Pi ? (\Tilde{v})^{ch}} S_1' $ and $ S_2 \xrightarrow[]{\Pi ? (\Tilde{v})^{ch}} S_2'$. By \texttt{Nsys} on $S_1=\langle \gamma , \mathsf{LS} \rangle : P_1$ and $S_2=\langle \gamma , \mathsf{LS} \rangle : P_2$, we get $\langle \gamma , \mathsf{LS} \rangle : P_1  \xmapsto[]{\widetilde{\Pi ? (\Tilde{v})^{ch}}} \langle \gamma , \mathsf{LS} \rangle : P_1$ and $\langle \gamma , \mathsf{LS} \rangle : P_2  \xmapsto[]{\widetilde{\Pi ? (\Tilde{v})^{ch}}} \langle \gamma , \mathsf{LS} \rangle : P_2$. This gives us $S_1 \ || \ S_2 \xrightarrow[]{\Pi ? (\Tilde{v})^{ch}} S_1 \ || \ S_2$. By \texttt{Prec} and \texttt{Nsys} on both $S_2$ and $S_1$, we also get $S_2 \ || \ S_1 \xrightarrow[]{\Pi ? (\Tilde{v})^{ch}} S_2' \ || \ S_1$ for $S_2\ ||\ S_1$, showing $S_1 \ || \ S_2 \equiv S_2\ ||\ S_1$.\\
\\
\textbf{Case 3: } Consider $\alpha = \Pi \mathsf{G}(\Tilde{v})^{ch}$.\\
\indent \textbf{Subcase 3.1: }$S_1 \ || \ S_2 \xrightarrow[]{\alpha} S_1' \ || \ S_2'$. Assume $S_1\not= S_1'$ and $S_2\not= S_2'$, as those cases are covered by subcases 1.2, 1.3 and 1.4. The rules which have $\Pi \mathsf{G}(\Tilde{v})^{ch}$ exposed are \texttt{Lget} and \texttt{Rget}. Each of these only transition one of the systems, while the other stays in the same state. This means it is not possible for $S_1\not= S_1'$ and $S_2\not= S_2'$. This makes sense, as only a single process can ask to get information at a time, just like only 1 send transition can be send at a time. If the second system also transitions, this would have to be by supplying information, however this would result in a hidden, $\tau$, transition. Therefore this case is not possible.\\
\indent \textbf{Subcase 3.2: }$S_1 \ || \ S_2 \xrightarrow[]{\alpha} S_1' \ || \ S_2$. In this case only $S_1$ transitions. \\
\indent \indent \textbf{Subcase 3.2.1: } Consider \texttt{Lget}, we get $S_1 \xrightarrow[]{\Pi \mathsf{G}(\Tilde{v})^{ch}} S_1'$, resulting in $S_1 \ || \ S_2 \xrightarrow[]{\Pi \mathsf{G}(\Tilde{v})^{ch}} S_1' \ || \ S_2$. By \texttt{Rget} on $S_2\ ||\ S_1$, we also get $S_1 \xrightarrow[]{\Pi \mathsf{G}(\Tilde{v})^{ch}} S_1'$, resulting in $S_2 \ || \ S_1 \xrightarrow[]{\Pi \mathsf{G}(\Tilde{v})^{ch}} S_2 \ || \ S_1'$, showing $S_1 \ || \ S_2 \equiv S_2\ ||\ S_1$ in this case.\\ 
\indent \indent \textbf{Subcase 3.2.2: } Consider \texttt{Rget}, this would require $S_2$ to transition, but we know it does not. So this subcase is not possible.\\
\indent \textbf{Subcase 3.3: }$S_1 \ || \ S_2 \xrightarrow[]{\alpha} S_1 \ || \ S_2'$. In this case we have the opposite of the previous case.\\
\indent \indent \textbf{Subcase 3.2.1: } Consider \texttt{Lget}, this would require $S_2$ to transition, but we know it does not. So this subcase is not possible.\\ 
\indent \indent \textbf{Subcase 3.2.2: } Consider \texttt{Rget}, we get $S_2 \xrightarrow[]{\Pi \mathsf{G}(\Tilde{v})^{ch}} S_2'$, resulting in $S_1 \ || \ S_2 \xrightarrow[]{\Pi \mathsf{G}(\Tilde{v})^{ch}} S_1 \ || \ S_2'$. By \texttt{Lget} on $S_2\ ||\ S_1$, we also get $S_2 \xrightarrow[]{\Pi \mathsf{G}(\Tilde{v})^{ch}} S_2'$, resulting in $S_2 \ || \ S_1 \xrightarrow[]{\Pi \mathsf{G}(\Tilde{v})^{ch}} S_2' \ || \ S_1$, showing $S_1 \ || \ S_2 \equiv S_2\ ||\ S_1$ in this case.\\
\indent \textbf{Subcase 3.4:} $S_1 \ || \ S_2 \xrightarrow[]{\alpha} S_1 \ || \ S_2$. Consider \texttt{Nsys}, is states the only transition which can be discarded is the send transition. As that is not our $\alpha$, this case is not possible.\\
\\
\textbf{Case 4: } Consider $\alpha = \Pi \mathsf{S}(\Tilde{v})^{ch}$.\\
\indent \textbf{Subcase 4.1: }$S_1 \ || \ S_2 \xrightarrow[]{\alpha} S_1' \ || \ S_2'$. Assume $S_1\not= S_1'$ and $S_2\not= S_2'$, as those cases are covered by subcases 1.2, 1.3 and 1.4. Just like in subcase 3.1, this case is not possible. Only 1 system can supply at a time, meaning we cannot have two systems transition with an exposed supply transition. \\
\indent \textbf{Subcase 4.2: }$S_1 \ || \ S_2 \xrightarrow[]{\alpha} S_1' \ || \ S_2$. In this case only $S_1$ transitions. \\ 
\indent \indent \textbf{Subcase 4.2.1: } Consider \texttt{Lsup}, we get $S_1 \xrightarrow[]{\Pi \mathsf{S}(\Tilde{v})^{ch}} S_1'$, resulting in $S_1 \ || \ S_2 \xrightarrow[]{\Pi \mathsf{S}(\Tilde{v})^{ch}} S_1' \ || \ S_2$. By \texttt{Rsup} on $S_2\ ||\ S_1$, we also get $S_1 \xrightarrow[]{\Pi \mathsf{S}(\Tilde{v})^{ch}} S_1'$, resulting in $S_2 \ || \ S_1 \xrightarrow[]{\Pi \mathsf{S}(\Tilde{v})^{ch}} S_2 \ || \ S_1'$, showing $S_1 \ || \ S_2 \equiv S_2\ ||\ S_1$ in this case.\\
\indent \indent \textbf{Subcase 4.2.2: } Consider \texttt{Rsup}, this would require $S_2$ to transition, but we know it does not. So this subcase is not possible.\\
\indent \textbf{Subcase 4.3: }$S_1 \ || \ S_2 \xrightarrow[]{\alpha} S_1 \ || \ S_2'$. In this case we have the opposite of the previous case.\\
\indent \indent \textbf{Subcase 4.3.1: } Consider \texttt{Lsup}, this would require $S_2$ to transition, but we know it does not. So this subcase is not possible.\\
\indent \indent \textbf{Subcase 4.3.2: } Consider \texttt{Rsup}, we get $S_2 \xrightarrow[]{\Pi \mathsf{S}(\Tilde{v})^{ch}} S_2'$, resulting in $S_1 \ || \ S_2 \xrightarrow[]{\Pi \mathsf{S}(\Tilde{v})^{ch}} S_1 \ || \ S_2'$. By \texttt{Lsup} on $S_2\ ||\ S_1$, we also get $S_2 \xrightarrow[]{\Pi \mathsf{S}(\Tilde{v})^{ch}} S_2'$, resulting in $S_2 \ || \ S_1 \xrightarrow[]{\Pi \mathsf{S}(\Tilde{v})^{ch}} S_2' \ || \ S_1$, showing $S_1 \ || \ S_2 \equiv S_2\ ||\ S_1$ in this case.\\
\indent \textbf{Subcase 4.4:} $S_1 \ || \ S_2 \xrightarrow[]{\alpha} S_1 \ || \ S_2$. Consider \texttt{Nsys}, is states the only transition which can be discarded is the send transition. As that is not our $\alpha$, this case is not possible.\\
\\
\textbf{Case 5: } Consider $\alpha = \tau$.\\
\indent \textbf{Subcase 5.1: }$S_1 \ || \ S_2 \xrightarrow[]{\alpha} S_1' \ || \ S_2'$. Assume $S_1\not= S_1'$ and $S_2\not= S_2'$, as those cases are covered by subcases 1.2, 1.3 and 1.4. This leaves us with 2 subcases, as \texttt{Ltau} and \texttt{Rtau} only have 1 system transitioning.\\
\indent \indent \textbf{Subcase 5.1.1: } Consider \texttt{Luni}, we get $S_1 \xrightarrow[]{\Pi \mathsf{G}(\Tilde{v})^{ch}} S_1'$ and $S_2 \xrightarrow[]{\Pi \mathsf{S}(\Tilde{v})^{ch}} S_2'$ resulting in $S_1 \ || \ S_2 \xrightarrow[]{\tau} S_1' \ || \ S_2'$. By \texttt{Runi}, we also get $S_2 \xrightarrow[]{\Pi \mathsf{S}(\Tilde{v})^{ch}} S_2'$ and $S_1 \xrightarrow[]{\Pi \mathsf{G}(\Tilde{v})^{ch}} S_1'$ for $S_2\ ||\ S_1$. Resulting in $S_2 \ || \ S_1 \xrightarrow[]{\tau} S_2' \ || \ S_1'$. Therefore $S_1 \ || \ S_2 \equiv S_2\ ||\ S_1$ holds for this case.\\
\indent \indent \textbf{Subcase 5.1.2: } Consider \texttt{Runi}, we get $S_1 \xrightarrow[]{\Pi \mathsf{S}(\Tilde{v})^{ch}} S_1'$ and $S_2 \xrightarrow[]{\Pi \mathsf{G}(\Tilde{v})^{ch}} S_2'$ resulting in $S_1 \ || \ S_2 \xrightarrow[]{\tau} S_1' \ || \ S_2'$. By \texttt{Luni}, we also get $S_2 \xrightarrow[]{\Pi \mathsf{G}(\Tilde{v})^{ch}} S_2'$ and $S_1 \xrightarrow[]{\Pi \mathsf{S}(\Tilde{v})^{ch}} S_1'$ for $S_2\ ||\ S_1$. Resulting in $S_2 \ || \ S_1 \xrightarrow[]{\tau} S_2' \ || \ S_1'$. Therefore $S_1 \ || \ S_2 \equiv S_2\ ||\ S_1$ holds for this case.\\
\indent \textbf{Subcase 5.2: }$S_1 \ || \ S_2 \xrightarrow[]{\alpha} S_1' \ || \ S_2$. In this case only $S_1$ transitions with $\tau$. Three of the $\tau$-transition rules do not allow this, because by \texttt{Nsys} we cannot discard the transition. The rules \texttt{Luni}, \texttt{Runi} and \texttt{Rtau} require $S_2$ to transition, which for our subcase is not the case. This leaves us with one more case. Consider \texttt{Ltau}, we get $S_1 \xrightarrow[]{\tau} S_1'$ resulting in $S_1 \ || \ S_2 \xrightarrow[]{\tau} S_1' \ || \ S_2$. By \texttt{Rtau}, also $S_2\ ||\ S_1$ gives us $S_1 \xrightarrow[]{\tau} S_1'$ and $S_2 \ || \ S_1 \xrightarrow[]{\tau} S_2 \ || \ S_1'$. Meaning $S_1 \ || \ S_2 \equiv S_2\ ||\ S_1$ holds for this case.\\ 
\indent \textbf{Subcase 5.3: }$S_1 \ || \ S_2 \xrightarrow[]{\alpha} S_1 \ || \ S_2'$. In this case we have the opposite of the previous case. Once again we have 3 transitions which are not possible, namely \texttt{Luni}, \texttt{Runi} and \texttt{Ltau}. Consider \texttt{Rtau}, we get $S_2 \xrightarrow[]{\tau} S_2'$ resulting in $S_1 \ || \ S_2 \xrightarrow[]{\tau} S_1 \ || \ S_2'$. By \texttt{Ltau}, also $S_2\ ||\ S_1$ gives us $S_2 \xrightarrow[]{\tau} S_2'$ and $S_2 \ || \ S_1 \xrightarrow[]{\tau} S_2' \ || \ S_1$. Meaning $S_1 \ || \ S_2 \equiv S_2\ ||\ S_1$ holds for this case.\\
\indent \textbf{Subcase 5.4:} $S_1 \ || \ S_2 \xrightarrow[]{\alpha} S_1 \ || \ S_2$. This case is not possible, for a $\tau$, transition to happen, we need at least 1 system to make that transition as by \texttt{Nsys}, this is not possible to discard.\\
As it holds for all cases, we have proven commutativity. $\square$
\\
\textbf{2. }We have shown commutativity and the next point is associativity, $(S_1 \ ||\ S_2 )\ || \ S_3 \equiv S_1 \ ||\ (S_2 \ || \ S_3)$. We will once again perform a case analysis to prove this. Because our rules only deal with parallel compositions of 2 systems, we will consider the two systems between parentheses as 1. By our grammer, we can then consider the possible other rules and match those.\\
\textbf{Case 1: } Consider $\alpha = \Pi ! (\Tilde{v})^{ch}$.\\
\indent \textbf{Subcase 1.1: }$(S_1 \ || \ S_2)\ ||\ S_3 \xrightarrow[]{\alpha} (S_1' \ || \ S_2')\ ||\ S_3'$. Assume $S_i\not = S_i'$ as that is covered by other subcases. As we have a sending transition, we will end up with 3 subcases, namely of each system being the sending system, because only 1 system can send at a time.\\
\indent \indent \textbf{Subcase 1.1.1: } Consider \texttt{Lpar}, we get $S_1 \ || \ S_2 \xrightarrow[]{\Pi ! (\Tilde{v})^{ch}} S_1' \ || \ S_2'$ and $S_3 \xrightarrow[]{\Pi ? (\Tilde{v})^{ch}} S_3'$. We can now split the first transition into two subcases, one where $S_1$ is sending and one where $S_2$ is sending.\\
\indent \indent \indent \textbf{Subcase 1.1.1.1: } Consider \texttt{Lpar} on $S_1\ ||\ S_2$. We get $S_1 \xrightarrow[]{\Pi ! (\Tilde{v})^{ch}} S_1'$ and $S_2 \xrightarrow[]{\Pi ? (\Tilde{v})^{ch}} S_2'$, together with $S_3 \xrightarrow[]{\Pi ? (\Tilde{v})^{ch}} S_3'$, we then have $(S_1 \ || \ S_2)\ ||\ S_3 \xrightarrow[]{\Pi ! (\Tilde{v})^{ch}} (S_1' \ || \ S_2')\ ||\ S_3'$. By \texttt{Lpar} on $S_1 \ ||\ (S_2 \ || \ S_3)$, we get $S_1 \xrightarrow[]{\Pi ! (\Tilde{v})^{ch}} S_1'$ and $S_2 \ || \ S_3 \xrightarrow[]{\Pi ? (\Tilde{v})^{ch}} S_2' \ || \ S_3'$. By \texttt{Prec}, we then get $S_2 \xrightarrow[]{\Pi ? (\Tilde{v})^{ch}} S_2'$ and $S_3 \xrightarrow[]{\Pi ? (\Tilde{v})^{ch}} S_3'$, which are the same transitions and shows $(S_1 \ ||\ S_2 )\ || \ S_3 \equiv S_1 \ ||\ (S_2 \ || \ S_3)$ for this case.\\
\indent \indent \indent \textbf{Subcase 1.1.1.2: } Now consider \texttt{Rpar} on $S_1\ ||\ S_2$. We get $S_1 \xrightarrow[]{\Pi ? (\Tilde{v})^{ch}} S_1'$ and $S_2 \xrightarrow[]{\Pi ! (\Tilde{v})^{ch}} S_2'$, together with $S_3 \xrightarrow[]{\Pi ? (\Tilde{v})^{ch}} S_3'$, we then have $(S_1 \ || \ S_2)\ ||\ S_3 \xrightarrow[]{\Pi ! (\Tilde{v})^{ch}} (S_1' \ || \ S_2')\ ||\ S_3'$. By \texttt{Rpar} on $S_1 \ ||\ (S_2 \ || \ S_3)$, we get $S_1 \xrightarrow[]{\Pi ? (\Tilde{v})^{ch}} S_1'$ and $S_2 \ || \ S_3 \xrightarrow[]{\Pi ! (\Tilde{v})^{ch}} S_2' \ || \ S_3'$. By \texttt{Lpar}, we then get $S_2 \xrightarrow[]{\Pi ! (\Tilde{v})^{ch}} S_2'$ and $S_3 \xrightarrow[]{\Pi ? (\Tilde{v})^{ch}} S_3'$, which are the same transitions and shows $(S_1 \ ||\ S_2 )\ || \ S_3 \equiv S_1 \ ||\ (S_2 \ || \ S_3)$ for this case.\\
\indent \indent \textbf{Subcase 1.1.2: } Consider \texttt{Rpar}, we get $S_1 \ || \ S_2 \xrightarrow[]{\Pi ? (\Tilde{v})^{ch}} S_1' \ || \ S_2'$ and $S_3 \xrightarrow[]{\Pi ! (\Tilde{v})^{ch}} S_3'$. By \texttt{Prec} on $S_1\ ||\ S_2$, we then get $S_1 \xrightarrow[]{\Pi ? (\Tilde{v})^{ch}} S_1'$ and $S_2 \xrightarrow[]{\Pi ? (\Tilde{v})^{ch}} S_2'$. By \texttt{Rpar} on $S_1 \ ||\ (S_2 \ || \ S_3)$, we get $S_1 \xrightarrow[]{\Pi ? (\Tilde{v})^{ch}} S_1'$ and $S_2 \ || \ S_3 \xrightarrow[]{\Pi ! (\Tilde{v})^{ch}} S_2' \ || \ S_3'$. By \texttt{Rpar} on $S_2\ ||\ S_3$, we then get $S_2 \xrightarrow[]{\Pi ? (\Tilde{v})^{ch}} S_2'$ and $S_3 \xrightarrow[]{\Pi ! (\Tilde{v})^{ch}} S_3'$, which are the same transitions and shows $(S_1 \ ||\ S_2 )\ || \ S_3 \equiv S_1 \ ||\ (S_2 \ || \ S_3)$ for this case.\\
\indent \textbf{Subcase 1.2: }$(S_1 \ || \ S_2)\ ||\ S_3 \xrightarrow[]{\alpha} (S_1' \ || \ S_2)\ ||\ S_3$. Here only $S_1$ transitions. From the commutativity proof, we know that for our $\alpha$, the sending transition, the sending system has to transition. This means our sending transition has to be $S_1$.\\
\indent \indent \textbf{Subcase 1.2.1: } Consider \texttt{Lpar}, we get $S_1 \ || \ S_2 \xrightarrow[]{\Pi ! (\Tilde{v})^{ch}} S_1' \ || \ S_2'$ and $S_3 \xrightarrow[]{\Pi ? (\Tilde{v})^{ch}} S_3'$.\\
\indent \indent \indent \textbf{Subcase 1.2.1.1: } Consider \texttt{Lpar} on $S_1\ ||\ S_2$. We get $S_1 \xrightarrow[]{\Pi ! (\Tilde{v})^{ch}} S_1'$ and $S_2 \xrightarrow[]{\Pi ? (\Tilde{v})^{ch}} S_2'$. By \texttt{Nsys} on $S_2 \xrightarrow[]{\Pi ? (\Tilde{v})^{ch}} S_2'$ and $S_3 \xrightarrow[]{\Pi ? (\Tilde{v})^{ch}} S_3'$, we get $S_2 \xrightarrow[]{\Pi ? (\Tilde{v})^{ch}} S_2$ and $S_3 \xrightarrow[]{\Pi ? (\Tilde{v})^{ch}} S_3$. We then have $(S_1 \ || \ S_2)\ ||\ S_3 \xrightarrow[]{\Pi ! (\Tilde{v})^{ch}} (S_1' \ || \ S_2)\ ||\ S_3$. By \texttt{Lpar} on $S_1 \ ||\ (S_2 \ || \ S_3)$, we get $S_1 \xrightarrow[]{\Pi ! (\Tilde{v})^{ch}} S_1'$ and $S_2 \ || \ S_3 \xrightarrow[]{\Pi ? (\Tilde{v})^{ch}} S_2' \ || \ S_3'$. By \texttt{Prec} and \texttt{Nsys} on both transitions, we then get $S_2 \xrightarrow[]{\Pi ? (\Tilde{v})^{ch}} S_2$ and $S_3 \xrightarrow[]{\Pi ? (\Tilde{v})^{ch}} S_3$, which are the same transitions and shows $(S_1 \ ||\ S_2 )\ || \ S_3 \equiv S_1 \ ||\ (S_2 \ || \ S_3)$ for this case. \\
\indent \indent \indent \textbf{Subcase 1.2.1.1: } Consider \texttt{Rpar} on $S_1 \ || \ S_2 \xrightarrow[]{\Pi ! (\Tilde{v})^{ch}} S_1' \ || \ S_2'$. This would require $S_2$ to be the sending system, however we know $S_2$ does not transition and thus this is not possible.\\
\indent \indent \textbf{Subcase 1.2.2: } Consider \texttt{Rpar}. This would require $S_3$ to be the sending system, however we know $S_3$ does not transition and thus this is not possible.\\
\indent \textbf{Subcase 1.3: }$(S_1 \ || \ S_2)\ ||\ S_3 \xrightarrow[]{\alpha} (S_1 \ || \ S_2')\ ||\ S_3$. This case is similar to the previous subcase, but this time we have $S_2$ which has to be the sending transition.\\
\indent \indent \textbf{Subcase 1.3.1: } Consider \texttt{Lpar}, we get $S_1 \ || \ S_2 \xrightarrow[]{\Pi ! (\Tilde{v})^{ch}} S_1' \ || \ S_2'$ and $S_3 \xrightarrow[]{\Pi ? (\Tilde{v})^{ch}} S_3'$.\\
\indent \indent \indent \textbf{Subcase 1.3.1.1: } Consider \texttt{Lpar} on $S_1 \ || \ S_2 \xrightarrow[]{\Pi ! (\Tilde{v})^{ch}} S_1' \ || \ S_2'$. This would require $S_1$ to be the sending system, however we know $S_1$ does not transition and thus this is not possible.\\
\indent \indent \indent \textbf{Subcase 1.3.1.2: } Consider \texttt{Rpar} on $S_1\ ||\ S_2$. We get $S_1 \xrightarrow[]{\Pi ? (\Tilde{v})^{ch}} S_1'$ and $S_2 \xrightarrow[]{\Pi ! (\Tilde{v})^{ch}} S_2'$. By \texttt{Nsys} on $S_1 \xrightarrow[]{\Pi ? (\Tilde{v})^{ch}} S_1'$ and $S_3 \xrightarrow[]{\Pi ? (\Tilde{v})^{ch}} S_3'$, we get $S_1 \xrightarrow[]{\Pi ? (\Tilde{v})^{ch}} S_1$ and $S_3 \xrightarrow[]{\Pi ? (\Tilde{v})^{ch}} S_3$. We then have $(S_1 \ || \ S_2)\ ||\ S_3 \xrightarrow[]{\Pi ! (\Tilde{v})^{ch}} (S_1 \ || \ S_2')\ ||\ S_3$. By \texttt{Rpar} on $S_1 \ ||\ (S_2 \ || \ S_3)$, we get $S_1 \xrightarrow[]{\Pi ? (\Tilde{v})^{ch}} S_1'$ and $S_2 \ || \ S_3 \xrightarrow[]{\Pi ! (\Tilde{v})^{ch}} S_2' \ || \ S_3'$. By \texttt{Lpar} and \texttt{Nsys} on $S_3$, we then get $S_2 \xrightarrow[]{\Pi ! (\Tilde{v})^{ch}} S_2'$ and $S_3 \xrightarrow[]{\Pi ? (\Tilde{v})^{ch}} S_3$, which are the same transitions and shows $(S_1 \ ||\ S_2 )\ || \ S_3 \equiv S_1 \ ||\ (S_2 \ || \ S_3)$ for this case.\\
\indent \indent \textbf{Subcase 1.3.2: } Consider \texttt{Rpar}. This would require $S_3$ to be the sending system, however we know $S_3$ does not transition and thus this is not possible.\\
\indent \textbf{Subcase 1.4: }$(S_1 \ || \ S_2)\ ||\ S_3 \xrightarrow[]{\alpha} (S_1 \ || \ S_2)\ ||\ S_3'$. This is once again a similar case, but this time with $S_3$ being the sending transition.\\
\indent \indent \textbf{Subcase 1.4.1: } Consider \texttt{Lpar}. This would require $S_1$ or $S_2$ to be the sending system, however we know neither transitions and thus this is not possible.\\
\indent \indent \textbf{Subcase 1.4.2: } Consider \texttt{Rpar}, we get $S_1 \ || \ S_2 \xrightarrow[]{\Pi ? (\Tilde{v})^{ch}} S_1' \ || \ S_2'$ and $S_3 \xrightarrow[]{\Pi ! (\Tilde{v})^{ch}} S_3'$. Now consider \texttt{Prec} on $S_1\ ||\ S_2$, we get $S_1 \xrightarrow[]{\Pi ? (\Tilde{v})^{ch}} S_1'$ and $S_2 \xrightarrow[]{\Pi ? (\Tilde{v})^{ch}} S_2'$. By \texttt{Nsys} on $S_1 \xrightarrow[]{\Pi ? (\Tilde{v})^{ch}} S_1'$ and $S_2 \xrightarrow[]{\Pi ? (\Tilde{v})^{ch}} S_2'$, we get $S_1 \xrightarrow[]{\Pi ? (\Tilde{v})^{ch}} S_1$ and $S_2 \xrightarrow[]{\Pi ? (\Tilde{v})^{ch}} S_2$. Resulting in $(S_1 \ || \ S_2)\ ||\ S_3 \xrightarrow[]{\Pi ! (\Tilde{v})^{ch}} (S_1 \ || \ S_2)\ ||\ S_3' $. By \texttt{Rpar} on $S_1 \ ||\ (S_2 \ || \ S_3)$ we get $S_1 \xrightarrow[]{\Pi ? (\Tilde{v})^{ch}} S_1'$ and $S_2 \ || \ S_3 \xrightarrow[]{\Pi ! (\Tilde{v})^{ch}} S_2' \ || \ S_3'$. Then by \texttt{Rpar} again on the latter transition we get $S_2 \xrightarrow[]{\Pi ? (\Tilde{v})^{ch}} S_2'$ and $S_3 \xrightarrow[]{\Pi ! (\Tilde{v})^{ch}} S_3'$. Resulting in $(S_1 \ || \ S_2)\ ||\ S_3 \xrightarrow[]{\Pi ! (\Tilde{v})^{ch}} (S_1 \ || \ S_2)\ ||\ S_3' $ with the same transitions. Showing $(S_1 \ ||\ S_2 )\ || \ S_3 \equiv S_1 \ ||\ (S_2 \ || \ S_3)$ for this case.\\
\indent \textbf{Subcase 1.5: }$(S_1 \ || \ S_2)\ ||\ S_3 \xrightarrow[]{\alpha} (S_1' \ || \ S_2')\ ||\ S_3$. In this case $S_1$ and $S_2$ transition, meaning either of them can be the sending system.\\
\indent \indent \textbf{Subcase 1.5.1: } Consider \texttt{Lpar}, we get $S_1 \ || \ S_2 \xrightarrow[]{\Pi ! (\Tilde{v})^{ch}} S_1' \ || \ S_2'$ and $S_3 \xrightarrow[]{\Pi ? (\Tilde{v})^{ch}} S_3'$.\\
\indent \indent \indent \textbf{Subcase 1.5.1.1: } Consider \texttt{Lpar} on $S_1\ ||\ S_2$. We get $S_1 \xrightarrow[]{\Pi ! (\Tilde{v})^{ch}} S_1'$ and $S_2 \xrightarrow[]{\Pi ? (\Tilde{v})^{ch}} S_2'$. By \texttt{Nsys} on $S_3 \xrightarrow[]{\Pi ? (\Tilde{v})^{ch}} S_3'$, we get $S_3 \xrightarrow[]{\Pi ? (\Tilde{v})^{ch}} S_3$, we then have $(S_1 \ || \ S_2)\ ||\ S_3 \xrightarrow[]{\Pi ! (\Tilde{v})^{ch}} (S_1' \ || \ S_2')\ ||\ S_3$. By \texttt{Lpar} on $S_1 \ ||\ (S_2 \ || \ S_3)$, we get $S_1 \xrightarrow[]{\Pi ! (\Tilde{v})^{ch}} S_1'$ and $S_2 \ || \ S_3 \xrightarrow[]{\Pi ? (\Tilde{v})^{ch}} S_2' \ || \ S_3'$. By \texttt{Prec}, we then get $S_2 \xrightarrow[]{\Pi ? (\Tilde{v})^{ch}} S_2'$ and $S_3 \xrightarrow[]{\Pi ? (\Tilde{v})^{ch}} S_3'$ and by \texttt{Nsys} on the latter we get $S_3 \xrightarrow[]{\Pi ? (\Tilde{v})^{ch}} S_3$, which are the same transitions and shows $(S_1 \ ||\ S_2 )\ || \ S_3 \equiv S_1 \ ||\ (S_2 \ || \ S_3)$ for this case.\\
\indent \indent \indent \textbf{Subcase 1.5.1.2: } Now consider \texttt{Rpar} on $S_1\ ||\ S_2$. We get $S_1 \xrightarrow[]{\Pi ? (\Tilde{v})^{ch}} S_1'$ and $S_2 \xrightarrow[]{\Pi ! (\Tilde{v})^{ch}} S_2'$. By \texttt{Nsys} on $S_3 \xrightarrow[]{\Pi ? (\Tilde{v})^{ch}} S_3'$, we get $S_3 \xrightarrow[]{\Pi ? (\Tilde{v})^{ch}} S_3$, we then have $(S_1 \ || \ S_2)\ ||\ S_3 \xrightarrow[]{\Pi ! (\Tilde{v})^{ch}} (S_1' \ || \ S_2')\ ||\ S_3$. By \texttt{Rpar} on $S_1 \ ||\ (S_2 \ || \ S_3)$, we get $S_1 \xrightarrow[]{\Pi ? (\Tilde{v})^{ch}} S_1'$ and $S_2 \ || \ S_3 \xrightarrow[]{\Pi ! (\Tilde{v})^{ch}} S_2' \ || \ S_3'$. By \texttt{Lpar}, we then get $S_2 \xrightarrow[]{\Pi ! (\Tilde{v})^{ch}} S_2'$ and $S_3 \xrightarrow[]{\Pi ? (\Tilde{v})^{ch}} S_3'$, and by \texttt{Nsys} we have $S_3 \xrightarrow[]{\Pi ? (\Tilde{v})^{ch}} S_3$, which are the same transitions and shows $(S_1 \ ||\ S_2 )\ || \ S_3 \equiv S_1 \ ||\ (S_2 \ || \ S_3)$ for this case.\\
\indent \indent \textbf{Subcase 1.5.2: } Consider \texttt{Rpar}, this would require $S_3$ to be the sending system, but as it does not transition this is not possible.\\
\indent \textbf{Subcase 1.6: }$(S_1 \ || \ S_2)\ ||\ S_3 \xrightarrow[]{\alpha} (S_1' \ || \ S_2)\ ||\ S_3'$. This is similar to the previous case, but now with $S_1$ or $S_3$ being the sending transition.\\
\indent \indent \textbf{Subcase 1.6.1: } Consider \texttt{Lpar}, we get $S_1 \ || \ S_2 \xrightarrow[]{\Pi ! (\Tilde{v})^{ch}} S_1' \ || \ S_2'$ and $S_3 \xrightarrow[]{\Pi ? (\Tilde{v})^{ch}} S_3'$.\\
\indent \indent \indent \textbf{Subcase 1.6.1.1: } Consider \texttt{Lpar} on $S_1\ ||\ S_2$. We get $S_1 \xrightarrow[]{\Pi ! (\Tilde{v})^{ch}} S_1'$ and $S_2 \xrightarrow[]{\Pi ? (\Tilde{v})^{ch}} S_2'$. By \texttt{Nsys} on $S_2 \xrightarrow[]{\Pi ? (\Tilde{v})^{ch}} S_2'$, we get $S_2 \xrightarrow[]{\Pi ? (\Tilde{v})^{ch}} S_2$. Together with $S_3 \xrightarrow[]{\Pi ? (\Tilde{v})^{ch}} S_3'$ we then have $(S_1 \ || \ S_2)\ ||\ S_3 \xrightarrow[]{\Pi ! (\Tilde{v})^{ch}} (S_1' \ || \ S_2)\ ||\ S_3'$. By \texttt{Lpar} on $S_1 \ ||\ (S_2 \ || \ S_3)$, we get $S_1 \xrightarrow[]{\Pi ! (\Tilde{v})^{ch}} S_1'$ and $S_2 \ || \ S_3 \xrightarrow[]{\Pi ? (\Tilde{v})^{ch}} S_2' \ || \ S_3'$. By \texttt{Prec}, we then get $S_2 \xrightarrow[]{\Pi ? (\Tilde{v})^{ch}} S_2'$ and $S_3 \xrightarrow[]{\Pi ? (\Tilde{v})^{ch}} S_3'$ and by \texttt{Nsys} on the latter we get $S_2 \xrightarrow[]{\Pi ? (\Tilde{v})^{ch}} S_2$, which are the same transitions and shows $(S_1 \ ||\ S_2 )\ || \ S_3 \equiv S_1 \ ||\ (S_2 \ || \ S_3)$ for this case.\\
\indent \indent \indent \textbf{Subcase 1.6.1.2: } Consider \texttt{Rpar} on $S_1\ ||\ S_2$. This would require $S_2$ to be the sending transition but as it does not transition this is not possible.\\
\indent \indent \textbf{Subcase 1.6.2: } Consider \texttt{Rpar}, we get $S_1 \ || \ S_2 \xrightarrow[]{\Pi ? (\Tilde{v})^{ch}} S_1' \ || \ S_2'$ and $S_3 \xrightarrow[]{\Pi ! (\Tilde{v})^{ch}} S_3'$. By \texttt{Prec} on $S_1\ ||\ S_2$, we then get $S_1 \xrightarrow[]{\Pi ? (\Tilde{v})^{ch}} S_1'$ and $S_2 \xrightarrow[]{\Pi ? (\Tilde{v})^{ch}} S_2'$, combined with \texttt{Nsys} on the latter we get $S_2 \xrightarrow[]{\Pi ? (\Tilde{v})^{ch}} S_2$. Resulting in $(S_1 \ || \ S_2)\ ||\ S_3 \xrightarrow[]{\Pi ! (\Tilde{v})^{ch}} (S_1' \ || \ S_2)\ ||\ S_3'$. By \texttt{Rpar} on $S_1 \ ||\ (S_2 \ || \ S_3)$, we get $S_1 \xrightarrow[]{\Pi ? (\Tilde{v})^{ch}} S_1'$ and $S_2 \ || \ S_3 \xrightarrow[]{\Pi ! (\Tilde{v})^{ch}} S_2' \ || \ S_3'$. By \texttt{Rpar} on $S_2\ ||\ S_3$, we then get $S_2 \xrightarrow[]{\Pi ? (\Tilde{v})^{ch}} S_2'$ and $S_3 \xrightarrow[]{\Pi ! (\Tilde{v})^{ch}} S_3'$. By \texttt{Nsys} on $S_2 \xrightarrow[]{\Pi ? (\Tilde{v})^{ch}} S_2'$ we get $S_2 \xrightarrow[]{\Pi ? (\Tilde{v})^{ch}} S_2$, which are the same transitions and shows $(S_1 \ ||\ S_2 )\ || \ S_3 \equiv S_1 \ ||\ (S_2 \ || \ S_3)$ for this case.\\
\indent \textbf{Subcase 1.7: }$(S_1 \ || \ S_2)\ ||\ S_3 \xrightarrow[]{\alpha} (S_1 \ || \ S_2')\ ||\ S_3'$. Once again this is a similar case to the previous subcases but with $S_2$ or $S_3$ being the sending system.\\
\indent \indent \textbf{Subcase 1.7.1: } Consider \texttt{Lpar}, we get $S_1 \ || \ S_2 \xrightarrow[]{\Pi ! (\Tilde{v})^{ch}} S_1' \ || \ S_2'$ and $S_3 \xrightarrow[]{\Pi ? (\Tilde{v})^{ch}} S_3'$.\\
\indent \indent \indent \textbf{Subcase 1.7.1.1: } Consider \texttt{Lpar} on $S_1\ ||\ S_2$. This would require $S_1$ to be the sending transition but as it does not transition this is not possible.\\ 
\indent \indent \indent \textbf{Subcase 1.7.1.2: }Consider \texttt{Rpar} on $S_1\ ||\ S_2$. We get $S_1 \xrightarrow[]{\Pi ? (\Tilde{v})^{ch}} S_1'$ and $S_2 \xrightarrow[]{\Pi ! (\Tilde{v})^{ch}} S_2'$. By \texttt{Nsys} on $S_1 \xrightarrow[]{\Pi ? (\Tilde{v})^{ch}} S_1'$, we get $S_1 \xrightarrow[]{\Pi ? (\Tilde{v})^{ch}} S_1$. Together with $S_3 \xrightarrow[]{\Pi ? (\Tilde{v})^{ch}} S_3'$ we then have $(S_1 \ || \ S_2)\ ||\ S_3 \xrightarrow[]{\Pi ! (\Tilde{v})^{ch}} (S_1 \ || \ S_2')\ ||\ S_3'$. By \texttt{Rpar} on $S_1 \ ||\ (S_2 \ || \ S_3)$, we get $S_1 \xrightarrow[]{\Pi ? (\Tilde{v})^{ch}} S_1'$ and $S_2 \ || \ S_3 \xrightarrow[]{\Pi ! (\Tilde{v})^{ch}} S_2' \ || \ S_3'$. By \texttt{Rpar} again, we then get $S_2 \xrightarrow[]{\Pi ! (\Tilde{v})^{ch}} S_2'$ and $S_3 \xrightarrow[]{\Pi ? (\Tilde{v})^{ch}} S_3'$ and by \texttt{Nsys} on $S_1 \xrightarrow[]{\Pi ? (\Tilde{v})^{ch}} S_1'$ we get $S_1 \xrightarrow[]{\Pi ? (\Tilde{v})^{ch}} S_1$. These are the same transitions and shows $(S_1 \ ||\ S_2 )\ || \ S_3 \equiv S_1 \ ||\ (S_2 \ || \ S_3)$ for this case.\\
\indent \indent \textbf{Subcase 1.7.2: }  Consider \texttt{Rpar}, we get $S_1 \ || \ S_2 \xrightarrow[]{\Pi ? (\Tilde{v})^{ch}} S_1' \ || \ S_2'$ and $S_3 \xrightarrow[]{\Pi ! (\Tilde{v})^{ch}} S_3'$. By \texttt{Prec} on $S_1\ ||\ S_2$, we then get $S_1 \xrightarrow[]{\Pi ? (\Tilde{v})^{ch}} S_1'$ and $S_2 \xrightarrow[]{\Pi ? (\Tilde{v})^{ch}} S_2'$, combined with \texttt{Nsys} on the first we get $S_1 \xrightarrow[]{\Pi ? (\Tilde{v})^{ch}} S_1$. Resulting in $(S_1 \ || \ S_2)\ ||\ S_3 \xrightarrow[]{\Pi ! (\Tilde{v})^{ch}} (S_1 \ || \ S_2')\ ||\ S_3'$. By \texttt{Rpar} on $S_1 \ ||\ (S_2 \ || \ S_3)$, we get $S_1 \xrightarrow[]{\Pi ? (\Tilde{v})^{ch}} S_1'$ and $S_2 \ || \ S_3 \xrightarrow[]{\Pi ! (\Tilde{v})^{ch}} S_2' \ || \ S_3'$. By \texttt{Rpar} on $S_2\ ||\ S_3$, we then get $S_2 \xrightarrow[]{\Pi ? (\Tilde{v})^{ch}} S_2'$ and $S_3 \xrightarrow[]{\Pi ! (\Tilde{v})^{ch}} S_3'$. By \texttt{Nsys} on $S_1 \xrightarrow[]{\Pi ? (\Tilde{v})^{ch}} S_1'$ we get $S_1 \xrightarrow[]{\Pi ? (\Tilde{v})^{ch}} S_1$, which are the same transitions and shows $(S_1 \ ||\ S_2 )\ || \ S_3 \equiv S_1 \ ||\ (S_2 \ || \ S_3)$ for this case.\\
\indent \textbf{Subcase 1.8: }$(S_1 \ || \ S_2)\ ||\ S_3 \xrightarrow[]{\alpha} (S_1 \ || \ S_2)\ ||\ S_3$. As none of the systems transition, this case is not possible as for a sending action, a system has to transition.\\
\\
\textbf{Case 2: } Consider $\alpha = \Pi ? (\Tilde{v})^{ch}$.\\
\indent \textbf{Subcase 2.1: }$(S_1 \ || \ S_2)\ ||\ S_3 \xrightarrow[]{\alpha} (S_1' \ || \ S_2')\ ||\ S_3'$. Consider \texttt{Prec}, we get $S_1 \ || \ S_2 \xrightarrow[]{\Pi ? (\Tilde{v})^{ch}} S_1' \ || \ S_2'$ and $S_3 \xrightarrow[]{\Pi ? (\Tilde{v})^{ch}} S_3'$. By \texttt{Prec} again on $S_1 \ || \ S_2 \xrightarrow[]{\Pi ? (\Tilde{v})^{ch}} S_1' \ || \ S_2'$, we get $S_1 \xrightarrow[]{\Pi ? (\Tilde{v})^{ch}} S_1'$ and $S_2 \xrightarrow[]{\Pi ? (\Tilde{v})^{ch}} S_2'$. This gives us $(S_1 \ || \ S_2)\ ||\ S_3 \xrightarrow[]{\Pi ? (\Tilde{v})^{ch}} (S_1' \ || \ S_2')\ ||\ S_3'$. By \texttt{Prec} twice on $S_1 \ ||\ (S_2 \ || \ S_3)$, we also get $S_1 \xrightarrow[]{\Pi ? (\Tilde{v})^{ch}} S_1'$, $S_2 \xrightarrow[]{\Pi ? (\Tilde{v})^{ch}} S_2'$ and $S_3 \xrightarrow[]{\Pi ? (\Tilde{v})^{ch}} S_3'$, which are the same transitions and shows $(S_1 \ ||\ S_2 )\ || \ S_3 \equiv S_1 \ ||\ (S_2 \ || \ S_3)$ for this case.\\
\indent \textbf{Subcase 2.2: }$(S_1 \ || \ S_2)\ ||\ S_3 \xrightarrow[]{\alpha} (S_1' \ || \ S_2)\ ||\ S_3$. Consider \texttt{Prec}, we get $S_1 \ || \ S_2 \xrightarrow[]{\Pi ? (\Tilde{v})^{ch}} S_1' \ || \ S_2'$ and $S_3 \xrightarrow[]{\Pi ? (\Tilde{v})^{ch}} S_3'$. By \texttt{Prec} again on $S_1 \ || \ S_2 \xrightarrow[]{\Pi ? (\Tilde{v})^{ch}} S_1' \ || \ S_2'$, we get $S_1 \xrightarrow[]{\Pi ? (\Tilde{v})^{ch}} S_1'$ and $S_2 \xrightarrow[]{\Pi ? (\Tilde{v})^{ch}} S_2'$. By \texttt{Nsys} on $S_2$ and $S_3$ we get $S_2 \xrightarrow[]{\Pi ? (\Tilde{v})^{ch}} S_2$ and $S_3 \xrightarrow[]{\Pi ? (\Tilde{v})^{ch}} S_3$. This gives us $(S_1 \ || \ S_2)\ ||\ S_3 \xrightarrow[]{\Pi ? (\Tilde{v})^{ch}} (S_1' \ || \ S_2)\ ||\ S_3$. By \texttt{Prec} twice on $S_1 \ ||\ (S_2 \ || \ S_3)$ and \texttt{Nsys} on the resulting $S_2$ and $S_3$ transitions, we also get $S_1 \xrightarrow[]{\Pi ? (\Tilde{v})^{ch}} S_1'$, $S_2 \xrightarrow[]{\Pi ? (\Tilde{v})^{ch}} S_2$ and $S_3 \xrightarrow[]{\Pi ? (\Tilde{v})^{ch}} S_3$, which are the same transitions and shows $(S_1 \ ||\ S_2 )\ || \ S_3 \equiv S_1 \ ||\ (S_2 \ || \ S_3)$ for this case.\\
\indent \textbf{Subcase 2.3: }$(S_1 \ || \ S_2)\ ||\ S_3 \xrightarrow[]{\alpha} (S_1 \ || \ S_2')\ ||\ S_3$. Consider \texttt{Prec}, we get $S_1 \ || \ S_2 \xrightarrow[]{\Pi ? (\Tilde{v})^{ch}} S_1' \ || \ S_2'$ and $S_3 \xrightarrow[]{\Pi ? (\Tilde{v})^{ch}} S_3'$. By \texttt{Prec} again on $S_1 \ || \ S_2 \xrightarrow[]{\Pi ? (\Tilde{v})^{ch}} S_1' \ || \ S_2'$, we get $S_1 \xrightarrow[]{\Pi ? (\Tilde{v})^{ch}} S_1'$ and $S_2 \xrightarrow[]{\Pi ? (\Tilde{v})^{ch}} S_2'$. By \texttt{Nsys} on $S_1$ and $S_3$ we get $S_1 \xrightarrow[]{\Pi ? (\Tilde{v})^{ch}} S_1$ and $S_3 \xrightarrow[]{\Pi ? (\Tilde{v})^{ch}} S_3$. This gives us $(S_1 \ || \ S_2)\ ||\ S_3 \xrightarrow[]{\Pi ? (\Tilde{v})^{ch}} (S_1 \ || \ S_2')\ ||\ S_3$. By \texttt{Prec} twice on $S_1 \ ||\ (S_2 \ || \ S_3)$ and \texttt{Nsys} on the resulting $S_1$ and $S_3$ transitions, we also get $S_1 \xrightarrow[]{\Pi ? (\Tilde{v})^{ch}} S_1$, $S_2 \xrightarrow[]{\Pi ? (\Tilde{v})^{ch}} S_2'$ and $S_3 \xrightarrow[]{\Pi ? (\Tilde{v})^{ch}} S_3$, which are the same transitions and shows $(S_1 \ ||\ S_2 )\ || \ S_3 \equiv S_1 \ ||\ (S_2 \ || \ S_3)$ for this case.\\
\indent \textbf{Subcase 2.4: }$(S_1 \ || \ S_2)\ ||\ S_3 \xrightarrow[]{\alpha} (S_1 \ || \ S_2)\ ||\ S_3'$. Consider \texttt{Prec}, we get $S_1 \ || \ S_2 \xrightarrow[]{\Pi ? (\Tilde{v})^{ch}} S_1' \ || \ S_2'$ and $S_3 \xrightarrow[]{\Pi ? (\Tilde{v})^{ch}} S_3'$. By \texttt{Prec} again on $S_1 \ || \ S_2 \xrightarrow[]{\Pi ? (\Tilde{v})^{ch}} S_1' \ || \ S_2'$, we get $S_1 \xrightarrow[]{\Pi ? (\Tilde{v})^{ch}} S_1'$ and $S_2 \xrightarrow[]{\Pi ? (\Tilde{v})^{ch}} S_2'$. By \texttt{Nsys} on $S_1$ and $S_2$ we get $S_1 \xrightarrow[]{\Pi ? (\Tilde{v})^{ch}} S_1$ and $S_2 \xrightarrow[]{\Pi ? (\Tilde{v})^{ch}} S_2$. This gives us $(S_1 \ || \ S_2)\ ||\ S_3 \xrightarrow[]{\Pi ? (\Tilde{v})^{ch}} (S_1 \ || \ S_2)\ ||\ S_3'$. By \texttt{Prec} twice on $S_1 \ ||\ (S_2 \ || \ S_3)$ and \texttt{Nsys} on the resulting $S_1$ and $S_2$ transitions, we also get $S_1 \xrightarrow[]{\Pi ? (\Tilde{v})^{ch}} S_1$, $S_2 \xrightarrow[]{\Pi ? (\Tilde{v})^{ch}} S_2$ and $S_3 \xrightarrow[]{\Pi ? (\Tilde{v})^{ch}} S_3'$, which are the same transitions and shows $(S_1 \ ||\ S_2 )\ || \ S_3 \equiv S_1 \ ||\ (S_2 \ || \ S_3)$ for this case.\\
\indent \textbf{Subcase 2.5: }$(S_1 \ || \ S_2)\ ||\ S_3 \xrightarrow[]{\alpha} (S_1' \ || \ S_2')\ ||\ S_3$. Consider \texttt{Prec}, we get $S_1 \ || \ S_2 \xrightarrow[]{\Pi ? (\Tilde{v})^{ch}} S_1' \ || \ S_2'$ and $S_3 \xrightarrow[]{\Pi ? (\Tilde{v})^{ch}} S_3'$. By \texttt{Prec} again on $S_1 \ || \ S_2 \xrightarrow[]{\Pi ? (\Tilde{v})^{ch}} S_1' \ || \ S_2'$, we get $S_1 \xrightarrow[]{\Pi ? (\Tilde{v})^{ch}} S_1'$ and $S_2 \xrightarrow[]{\Pi ? (\Tilde{v})^{ch}} S_2'$. By \texttt{Nsys} on $S_3$ we get $S_3 \xrightarrow[]{\Pi ? (\Tilde{v})^{ch}} S_3$. This gives us $(S_1 \ || \ S_2)\ ||\ S_3 \xrightarrow[]{\Pi ? (\Tilde{v})^{ch}} (S_1' \ || \ S_2')\ ||\ S_3$. By \texttt{Prec} twice on $S_1 \ ||\ (S_2 \ || \ S_3)$ and \texttt{Nsys} on the resulting $S_3$ transition, we also get $S_1 \xrightarrow[]{\Pi ? (\Tilde{v})^{ch}} S_1'$, $S_2 \xrightarrow[]{\Pi ? (\Tilde{v})^{ch}} S_2'$ and $S_3 \xrightarrow[]{\Pi ? (\Tilde{v})^{ch}} S_3$, which are the same transitions and shows $(S_1 \ ||\ S_2 )\ || \ S_3 \equiv S_1 \ ||\ (S_2 \ || \ S_3)$ for this case.\\
\indent \textbf{Subcase 2.6: }$(S_1 \ || \ S_2)\ ||\ S_3 \xrightarrow[]{\alpha} (S_1' \ || \ S_2)\ ||\ S_3'$. Consider \texttt{Prec}, we get $S_1 \ || \ S_2 \xrightarrow[]{\Pi ? (\Tilde{v})^{ch}} S_1' \ || \ S_2'$ and $S_3 \xrightarrow[]{\Pi ? (\Tilde{v})^{ch}} S_3'$. By \texttt{Prec} again on $S_1 \ || \ S_2 \xrightarrow[]{\Pi ? (\Tilde{v})^{ch}} S_1' \ || \ S_2'$, we get $S_1 \xrightarrow[]{\Pi ? (\Tilde{v})^{ch}} S_1'$ and $S_2 \xrightarrow[]{\Pi ? (\Tilde{v})^{ch}} S_2'$. By \texttt{Nsys} on $S_2$ we get $S_2 \xrightarrow[]{\Pi ? (\Tilde{v})^{ch}} S_2$. This gives us $(S_1 \ || \ S_2)\ ||\ S_3 \xrightarrow[]{\Pi ? (\Tilde{v})^{ch}} (S_1' \ || \ S_2)\ ||\ S_3'$. By \texttt{Prec} twice on $S_1 \ ||\ (S_2 \ || \ S_3)$ and \texttt{Nsys} on the resulting $S_2$ transition, we also get $S_1 \xrightarrow[]{\Pi ? (\Tilde{v})^{ch}} S_1'$, $S_2 \xrightarrow[]{\Pi ? (\Tilde{v})^{ch}} S_2$ and $S_3 \xrightarrow[]{\Pi ? (\Tilde{v})^{ch}} S_3'$, which are the same transitions and shows $(S_1 \ ||\ S_2 )\ || \ S_3 \equiv S_1 \ ||\ (S_2 \ || \ S_3)$ for this case.\\
\indent \textbf{Subcase 2.7: }$(S_1 \ || \ S_2)\ ||\ S_3 \xrightarrow[]{\alpha} (S_1 \ || \ S_2')\ ||\ S_3'$. Consider \texttt{Prec}, we get $S_1 \ || \ S_2 \xrightarrow[]{\Pi ? (\Tilde{v})^{ch}} S_1' \ || \ S_2'$ and $S_3 \xrightarrow[]{\Pi ? (\Tilde{v})^{ch}} S_3'$. By \texttt{Prec} again on $S_1 \ || \ S_2 \xrightarrow[]{\Pi ? (\Tilde{v})^{ch}} S_1' \ || \ S_2'$, we get $S_1 \xrightarrow[]{\Pi ? (\Tilde{v})^{ch}} S_1'$ and $S_2 \xrightarrow[]{\Pi ? (\Tilde{v})^{ch}} S_2'$. By \texttt{Nsys} on $S_1$ we get $S_1 \xrightarrow[]{\Pi ? (\Tilde{v})^{ch}} S_1$. This gives us $(S_1 \ || \ S_2)\ ||\ S_3 \xrightarrow[]{\Pi ? (\Tilde{v})^{ch}} (S_1 \ || \ S_2')\ ||\ S_3'$. By \texttt{Prec} twice on $S_1 \ ||\ (S_2 \ || \ S_3)$ and \texttt{Nsys} on the resulting $S_1$ transition, we also get $S_1 \xrightarrow[]{\Pi ? (\Tilde{v})^{ch}} S_1$, $S_2 \xrightarrow[]{\Pi ? (\Tilde{v})^{ch}} S_2'$ and $S_3 \xrightarrow[]{\Pi ? (\Tilde{v})^{ch}} S_3'$, which are the same transitions and shows $(S_1 \ ||\ S_2 )\ || \ S_3 \equiv S_1 \ ||\ (S_2 \ || \ S_3)$ for this case.\\
\indent \textbf{Subcase 2.8: }$(S_1 \ || \ S_2)\ ||\ S_3 \xrightarrow[]{\alpha} (S_1 \ || \ S_2)\ ||\ S_3$. Consider \texttt{Prec}, we get $S_1 \ || \ S_2 \xrightarrow[]{\Pi ? (\Tilde{v})^{ch}} S_1' \ || \ S_2'$ and $S_3 \xrightarrow[]{\Pi ? (\Tilde{v})^{ch}} S_3'$. By \texttt{Prec} again on $S_1 \ || \ S_2 \xrightarrow[]{\Pi ? (\Tilde{v})^{ch}} S_1' \ || \ S_2'$, we get $S_1 \xrightarrow[]{\Pi ? (\Tilde{v})^{ch}} S_1'$ and $S_2 \xrightarrow[]{\Pi ? (\Tilde{v})^{ch}} S_2'$. By \texttt{Nsys} on all three transitions we get $S_1 \xrightarrow[]{\Pi ? (\Tilde{v})^{ch}} S_1$, $S_2 \xrightarrow[]{\Pi ? (\Tilde{v})^{ch}} S_2$ and $S_3 \xrightarrow[]{\Pi ? (\Tilde{v})^{ch}} S_3$. This gives us $(S_1 \ || \ S_2)\ ||\ S_3 \xrightarrow[]{\Pi ? (\Tilde{v})^{ch}} (S_1 \ || \ S_2)\ ||\ S_3$. By \texttt{Prec} twice on $S_1 \ ||\ (S_2 \ || \ S_3)$ and \texttt{Nsys} on the resulting transitions, we also get $S_1 \xrightarrow[]{\Pi ? (\Tilde{v})^{ch}} S_1$, $S_2 \xrightarrow[]{\Pi ? (\Tilde{v})^{ch}} S_2$ and $S_3 \xrightarrow[]{\Pi ? (\Tilde{v})^{ch}} S_3$, which are the same transitions and shows $(S_1 \ ||\ S_2 )\ || \ S_3 \equiv S_1 \ ||\ (S_2 \ || \ S_3)$ for this case.\\
\\
\textbf{Case 3: } Consider $\alpha = \Pi \mathsf{G}(\Tilde{v})^{ch}$.\\
\indent \textbf{Subcase 3.1: }$(S_1 \ || \ S_2)\ ||\ S_3 \xrightarrow[]{\alpha} (S_1' \ || \ S_2')\ ||\ S_3'$. We know only one system can send a get request at a time. We also know that if a system supplies, the transition is hidden, this means for this $\alpha$ only 1 system can transition and thus this subcase is not possible.\\
\indent \textbf{Subcase 3.2: }$(S_1 \ || \ S_2)\ ||\ S_3 \xrightarrow[]{\alpha} (S_1' \ || \ S_2)\ ||\ S_3$. As we know a system has to transition to send a get request, we know $S_1$ has to be the getting system. The only way to get this is by applying \texttt{Lget} twice. We get $S_1\xrightarrow[]{\Pi \mathsf{G}(\Tilde{v})^{ch}} S_1'$. By \texttt{Lget} on $S_1 \ ||\ (S_2 \ || \ S_3)$, we also get $S_1\xrightarrow[]{\Pi \mathsf{G}(\Tilde{v})^{ch}} S_1'$. which is the same transition and shows $(S_1 \ ||\ S_2 )\ || \ S_3 \equiv S_1 \ ||\ (S_2 \ || \ S_3)$ for this case.\\
\indent \textbf{Subcase 3.3: }$(S_1 \ || \ S_2)\ ||\ S_3 \xrightarrow[]{\alpha} (S_1 \ || \ S_2')\ ||\ S_3$. As we know a system has to transition to send a get request, we know $S_2$ has to be the getting system. The only way to get this is by applying \texttt{Lget} and \texttt{Rget}. We get $S_2\xrightarrow[]{\Pi \mathsf{G}(\Tilde{v})^{ch}} S_2'$. By \texttt{Rget} and \texttt{Lget} on $S_1 \ ||\ (S_2 \ || \ S_3)$, we also get $S_2\xrightarrow[]{\Pi \mathsf{G}(\Tilde{v})^{ch}} S_2'$. which is the same transition and shows $(S_1 \ ||\ S_2 )\ || \ S_3 \equiv S_1 \ ||\ (S_2 \ || \ S_3)$ for this case.\\
\indent \textbf{Subcase 3.4: }$(S_1 \ || \ S_2)\ ||\ S_3 \xrightarrow[]{\alpha} (S_1 \ || \ S_2)\ ||\ S_3'$. As we know a system has to transition to send a get request, we know $S_3$ has to be the getting system. The only way to get this is by applying \texttt{Rget}. We get $S_3\xrightarrow[]{\Pi \mathsf{G}(\Tilde{v})^{ch}} S_3'$. By \texttt{Rget} twice on $S_1 \ ||\ (S_2 \ || \ S_3)$, we also get $S_3\xrightarrow[]{\Pi \mathsf{G}(\Tilde{v})^{ch}} S_3'$. which is the same transition and shows $(S_1 \ ||\ S_2 )\ || \ S_3 \equiv S_1 \ ||\ (S_2 \ || \ S_3)$ for this case.\\
\indent \textbf{Subcase 3.5: }$(S_1 \ || \ S_2)\ ||\ S_3 \xrightarrow[]{\alpha} (S_1' \ || \ S_2')\ ||\ S_3$. We know only one system can send a get request at a time. We also know that if a system supplies, the transition is hidden, this means for this $\alpha$ only 1 system can transition and thus this subcase is not possible.\\
\indent \textbf{Subcase 3.6: }$(S_1 \ || \ S_2)\ ||\ S_3 \xrightarrow[]{\alpha} (S_1' \ || \ S_2)\ ||\ S_3'$. We know only one system can send a get request at a time. We also know that if a system supplies, the transition is hidden, this means for this $\alpha$ only 1 system can transition and thus this subcase is not possible.\\
\indent \textbf{Subcase 3.7: }$(S_1 \ || \ S_2)\ ||\ S_3 \xrightarrow[]{\alpha} (S_1 \ || \ S_2')\ ||\ S_3'$. We know only one system can send a get request at a time. We also know that if a system supplies, the transition is hidden, this means for this $\alpha$ only 1 system can transition and thus this subcase is not possible.\\
\indent \textbf{Subcase 3.8: }$(S_1 \ || \ S_2)\ ||\ S_3 \xrightarrow[]{\alpha} (S_1 \ || \ S_2)\ ||\ S_3$. We know by \texttt{Nsys}, that only a receive transition can be discarded. This means for our get transition, at least 1 system has to transition and thus this subcase is not possible.\\
\\
\textbf{Case 4: } Consider $\alpha = \Pi \mathsf{S}(\Tilde{v})^{ch}$.\\
\indent \textbf{Subcase 4.1: }$(S_1 \ || \ S_2)\ ||\ S_3 \xrightarrow[]{\alpha} (S_1' \ || \ S_2')\ ||\ S_3'$. This case has similar reasoning to the previous case. We know only one system can supply at a time. We also know that if a system in the same parallel composition has sent a get request, the transition is hidden, this means for this $\alpha$ only 1 system can transition and thus this subcase is not possible.\\
\indent \textbf{Subcase 4.2: }$(S_1 \ || \ S_2)\ ||\ S_3 \xrightarrow[]{\alpha} (S_1' \ || \ S_2)\ ||\ S_3$. As we know a system has to transition to supply, we know $S_1$ has to be the supplying system. The only way to get this is by applying \texttt{Lsup} twice. We get $S_1\xrightarrow[]{\Pi \mathsf{S}(\Tilde{v})^{ch}} S_1'$. By \texttt{Lsup} on $S_1 \ ||\ (S_2 \ || \ S_3)$, we also get $S_1\xrightarrow[]{\Pi \mathsf{S}(\Tilde{v})^{ch}} S_1'$. This is the same transition and shows $(S_1 \ ||\ S_2 )\ || \ S_3 \equiv S_1 \ ||\ (S_2 \ || \ S_3)$ for this case.\\
\indent \textbf{Subcase 4.3: }$(S_1 \ || \ S_2)\ ||\ S_3 \xrightarrow[]{\alpha} (S_1 \ || \ S_2')\ ||\ S_3$. As we know a system has to transition to supply, we know $S_2$ has to be the supplying system. The only way to get this is by applying \texttt{Lsup} and \texttt{Rsup}. We get $S_2\xrightarrow[]{\Pi \mathsf{S}(\Tilde{v})^{ch}} S_2'$. By \texttt{Rsup} and \texttt{Lsup} on $S_1 \ ||\ (S_2 \ || \ S_3)$, we also get $S_2\xrightarrow[]{\Pi \mathsf{S}(\Tilde{v})^{ch}} S_2'$. This is the same transition and shows $(S_1 \ ||\ S_2 )\ || \ S_3 \equiv S_1 \ ||\ (S_2 \ || \ S_3)$ for this case.\\
\indent \textbf{Subcase 4.4: }$(S_1 \ || \ S_2)\ ||\ S_3 \xrightarrow[]{\alpha} (S_1 \ || \ S_2)\ ||\ S_3'$. As we know a system has to transition to supply, we know $S_3$ has to be the supplying system. The only way to get this is by applying \texttt{Rsup}. We get $S_3\xrightarrow[]{\Pi \mathsf{S}(\Tilde{v})^{ch}} S_3'$. By \texttt{Rsup} twice on $S_1 \ ||\ (S_2 \ || \ S_3)$, we also get $S_3\xrightarrow[]{\Pi \mathsf{S}(\Tilde{v})^{ch}} S_3'$. This is the same transition and shows $(S_1 \ ||\ S_2 )\ || \ S_3 \equiv S_1 \ ||\ (S_2 \ || \ S_3)$ for this case.\\
\indent \textbf{Subcase 4.5: }$(S_1 \ || \ S_2)\ ||\ S_3 \xrightarrow[]{\alpha} (S_1' \ || \ S_2')\ ||\ S_3$. We know only one system can supply at a time. We also know that if a system in the same parallel composition has sent a get request, the transition is hidden, this means for this $\alpha$ only 1 system can transition and thus this subcase is not possible.\\
\indent \textbf{Subcase 4.6: }$(S_1 \ || \ S_2)\ ||\ S_3 \xrightarrow[]{\alpha} (S_1' \ || \ S_2)\ ||\ S_3'$. We know only one system can supply at a time. We also know that if a system in the same parallel composition has sent a get request, the transition is hidden, this means for this $\alpha$ only 1 system can transition and thus this subcase is not possible.\\
\indent \textbf{Subcase 4.7: }$(S_1 \ || \ S_2)\ ||\ S_3 \xrightarrow[]{\alpha} (S_1 \ || \ S_2')\ ||\ S_3'$. We know only one system can supply at a time. We also know that if a system in the same parallel composition has sent a get request, the transition is hidden, this means for this $\alpha$ only 1 system can transition and thus this subcase is not possible.\\
\indent \textbf{Subcase 4.8: }$(S_1 \ || \ S_2)\ ||\ S_3 \xrightarrow[]{\alpha} (S_1 \ || \ S_2)\ ||\ S_3$ We know by \texttt{Nsys}, that only a receive transition can be discarded. This means for our supply transition, at least 1 system has to transition and thus this subcase is not possible.\\
\\
\textbf{Case 5: } Consider $\alpha = \tau$.\\
\indent \textbf{Subcase 1.1: }$(S_1 \ || \ S_2)\ ||\ S_3 \xrightarrow[]{\alpha} (S_1' \ || \ S_2')\ ||\ S_3'$. For a $\tau$ transition, at most two systems can transition, the getting and the supplying one. As we have three systems transition for this subcase, this is not possible.\\
\indent \textbf{Subcase 5.2: }$(S_1 \ || \ S_2)\ ||\ S_3 \xrightarrow[]{\alpha} (S_1' \ || \ S_2)\ ||\ S_3$. For a $\tau$ transition, we need a get and a supply transition. As we only have one transitioning system, $S_1$, we know this one has to do both and thus we have a hidden transition. The only way this is possible is by \texttt{Ltau} twice. We get $S_1\xrightarrow[]{\tau} S_1'$, resulting in $(S_1 \ || \ S_2)\ ||\ S_3 \xrightarrow[]{\tau} (S_1' \ || \ S_2)\ ||\ S_3$. By \texttt{Ltau}, we also get $S_1\xrightarrow[]{\tau} S_1'$ for $S_1 \ ||\ (S_2 \ || \ S_3)$. This is the same transition and shows $(S_1 \ ||\ S_2 )\ || \ S_3 \equiv S_1 \ ||\ (S_2 \ || \ S_3)$ for this case.\\
\indent \textbf{Subcase 5.3: }$(S_1 \ || \ S_2)\ ||\ S_3 \xrightarrow[]{\alpha} (S_1 \ || \ S_2')\ ||\ S_3$. For a $\tau$ transition, we need a get and a supply transition. As we only have one transitioning system, $S_2$, we know this one has to do both and thus we have a hidden transition. The only way this is possible is by \texttt{Ltau} and \texttt{Rtau}. We get $S_2\xrightarrow[]{\tau} S_2'$, resulting in $(S_1 \ || \ S_2)\ ||\ S_3 \xrightarrow[]{\tau} (S_1 \ || \ S_2')\ ||\ S_3$. By \texttt{Rtau} and \texttt{Ltau}, we also get $S_2\xrightarrow[]{\tau} S_2'$ for $S_1 \ ||\ (S_2 \ || \ S_3)$. This is the same transition and shows $(S_1 \ ||\ S_2 )\ || \ S_3 \equiv S_1 \ ||\ (S_2 \ || \ S_3)$ for this case.\\
\indent \textbf{Subcase 5.4: }$(S_1 \ || \ S_2)\ ||\ S_3 \xrightarrow[]{\alpha} (S_1 \ || \ S_2)\ ||\ S_3'$. For a $\tau$ transition, we need a get and a supply transition. As we only have one transitioning system, $S_3$, we know this one has to do both and thus we have a hidden transition. The only way this is possible is by \texttt{Rtau}. We get $S_3\xrightarrow[]{\tau} S_3'$, resulting in $(S_1 \ || \ S_2)\ ||\ S_3 \xrightarrow[]{\tau} (S_1 \ || \ S_2)\ ||\ S_3'$. By \texttt{Rtau} twice, we also get $S_3\xrightarrow[]{\tau} S_3'$ for $S_1 \ ||\ (S_2 \ || \ S_3)$. This is the same transition and shows $(S_1 \ ||\ S_2 )\ || \ S_3 \equiv S_1 \ ||\ (S_2 \ || \ S_3)$ for this case.\\
\indent \textbf{Subcase 5.5: }$(S_1 \ || \ S_2)\ ||\ S_3 \xrightarrow[]{\alpha} (S_1' \ || \ S_2')\ ||\ S_3$. For a $\tau$ transition at most two systems can transition. One which send a get request and one which supplies. As we have two transitioning systems, $S_1$ and $S_2$, we know one has to supply and one has to get, giving us two subcases.\\
\indent \indent \textbf{Subcase 5.5.1: } Consider $S_1$ being the getting system, meaning $S_2$ supplies. Consider \texttt{Ltau} we get $S_1 \ || \ S_2 \xrightarrow[]{\tau} S_1' \ || \ S_2'$. By \texttt{Luni}, we then get $S_1\xrightarrow[]{\Pi \mathsf{G}(\Tilde{v})^{ch}} S_1'$ and $S_2\xrightarrow[]{\Pi \mathsf{S}(\Tilde{v})^{ch}} S_2'$, resulting in $(S_1 \ || \ S_2)\ ||\ S_3 \xrightarrow[]{\tau} (S_1' \ || \ S_2')\ ||\ S_3$. By \texttt{Luni} and \texttt{Lsup}, we also get $S_1\xrightarrow[]{\Pi \mathsf{G}(\Tilde{v})^{ch}} S_1'$ and $S_2\xrightarrow[]{\Pi \mathsf{S}(\Tilde{v})^{ch}} S_2'$ for $S_1 \ ||\ (S_2 \ || \ S_3)$. These are the same transitions and shows $(S_1 \ ||\ S_2 )\ || \ S_3 \equiv S_1 \ ||\ (S_2 \ || \ S_3)$ for this case.\\
\indent \indent \textbf{Subcase 5.5.2: } Consider $S_2$ being the getting system, meaning $S_1$ supplies. Consider \texttt{Ltau} we get $S_1 \ || \ S_2 \xrightarrow[]{\tau} S_1' \ || \ S_2'$. By \texttt{Runi}, we then get $S_1\xrightarrow[]{\Pi \mathsf{S}(\Tilde{v})^{ch}} S_1'$ and $S_2\xrightarrow[]{\Pi \mathsf{G}(\Tilde{v})^{ch}} S_2'$, resulting in $(S_1 \ || \ S_2)\ ||\ S_3 \xrightarrow[]{\tau} (S_1' \ || \ S_2')\ ||\ S_3$. By \texttt{Runi} and \texttt{Lget}, we also get $S_1\xrightarrow[]{\Pi \mathsf{S}(\Tilde{v})^{ch}} S_1'$ and $S_2\xrightarrow[]{\Pi \mathsf{G}(\Tilde{v})^{ch}} S_2'$ for $S_1 \ ||\ (S_2 \ || \ S_3)$. These are the same transitions and shows $(S_1 \ ||\ S_2 )\ || \ S_3 \equiv S_1 \ ||\ (S_2 \ || \ S_3)$ for this case.\\
\indent \textbf{Subcase 5.6: }$(S_1 \ || \ S_2)\ ||\ S_3 \xrightarrow[]{\alpha} (S_1' \ || \ S_2)\ ||\ S_3'$. For a $\tau$ transition at most two systems can transition. One which send a get request and one which supplies. As we have two transitioning systems, $S_1$ and $S_3$, we know one has to supply and one has to get, giving us two subcases.\\
\indent \indent \textbf{Subcase 5.6.1: } Consider $S_1$ being the getting system, meaning $S_3$ supplies. Consider \texttt{Luni} and \texttt{Lget} we get $S_1\xrightarrow[]{\Pi \mathsf{G}(\Tilde{v})^{ch}} S_1'$ and $S_3\xrightarrow[]{\Pi \mathsf{S}(\Tilde{v})^{ch}} S_3'$, resulting in $(S_1 \ || \ S_2)\ ||\ S_3 \xrightarrow[]{\tau} (S_1' \ || \ S_2)\ ||\ S_3'$. By \texttt{Luni} and \texttt{Rsup}, we also get $S_1\xrightarrow[]{\Pi \mathsf{G}(\Tilde{v})^{ch}} S_1'$ and $S_3\xrightarrow[]{\Pi \mathsf{S}(\Tilde{v})^{ch}} S_3'$ for $S_1 \ ||\ (S_2 \ || \ S_3)$. These are the same transitions and shows $(S_1 \ ||\ S_2 )\ || \ S_3 \equiv S_1 \ ||\ (S_2 \ || \ S_3)$ for this case.\\
\indent \indent \textbf{Subcase 5.6.2: } Consider $S_3$ being the getting system, meaning $S_1$ supplies. Consider \texttt{Runi} and \texttt{Lsup} we get $S_1\xrightarrow[]{\Pi \mathsf{S}(\Tilde{v})^{ch}} S_1'$ and $S_3\xrightarrow[]{\Pi \mathsf{G}(\Tilde{v})^{ch}} S_3'$, resulting in $(S_1 \ || \ S_2)\ ||\ S_3 \xrightarrow[]{\tau} (S_1' \ || \ S_2)\ ||\ S_3'$. By \texttt{Runi} and \texttt{Rget}, we also get $S_1\xrightarrow[]{\Pi \mathsf{S}(\Tilde{v})^{ch}} S_1'$ and $S_3\xrightarrow[]{\Pi \mathsf{G}(\Tilde{v})^{ch}} S_3'$ for $S_1 \ ||\ (S_2 \ || \ S_3)$. These are the same transitions and shows $(S_1 \ ||\ S_2 )\ || \ S_3 \equiv S_1 \ ||\ (S_2 \ || \ S_3)$ for this case.\\
\indent \textbf{Subcase 5.7: }$(S_1 \ || \ S_2)\ ||\ S_3 \xrightarrow[]{\alpha} (S_1 \ || \ S_2')\ ||\ S_3'$. For a $\tau$ transition at most two systems can transition. One which send a get request and one which supplies. As we have two transitioning systems, $S_2$ and $S_3$, we know one has to supply and one has to get, giving us two subcases.\\
\indent \indent \textbf{Subcase 5.7.1: } Consider $S_2$ being the getting system, meaning $S_3$ supplies. Consider \texttt{Luni} and \texttt{Rget} we get $S_2\xrightarrow[]{\Pi \mathsf{G}(\Tilde{v})^{ch}} S_2'$ and $S_3\xrightarrow[]{\Pi \mathsf{S}(\Tilde{v})^{ch}} S_3'$, resulting in $(S_1 \ || \ S_2)\ ||\ S_3 \xrightarrow[]{\tau} (S_1 \ || \ S_2')\ ||\ S_3'$. By \texttt{Rtau} and \texttt{Luni}, we also get $S_2\xrightarrow[]{\Pi \mathsf{G}(\Tilde{v})^{ch}} S_2'$ and $S_3\xrightarrow[]{\Pi \mathsf{S}(\Tilde{v})^{ch}} S_3'$ for $S_1 \ ||\ (S_2 \ || \ S_3)$. These are the same transitions and shows $(S_1 \ ||\ S_2 )\ || \ S_3 \equiv S_1 \ ||\ (S_2 \ || \ S_3)$ for this case.\\
\indent \indent \textbf{Subcase 5.7.2: } Consider $S_3$ being the getting system, meaning $S_2$ supplies. Consider \texttt{Runi} and \texttt{Rsup} we get $S_2\xrightarrow[]{\Pi \mathsf{S}(\Tilde{v})^{ch}} S_2'$ and $S_3\xrightarrow[]{\Pi \mathsf{G}(\Tilde{v})^{ch}} S_3'$, resulting in $(S_1 \ || \ S_2)\ ||\ S_3 \xrightarrow[]{\tau} (S_1 \ || \ S_2')\ ||\ S_3'$. By \texttt{Rtau} and \texttt{Runi}, we also get $S_2\xrightarrow[]{\Pi \mathsf{S}(\Tilde{v})^{ch}} S_2'$ and $S_3\xrightarrow[]{\Pi \mathsf{G}(\Tilde{v})^{ch}} S_3'$ for $S_1 \ ||\ (S_2 \ || \ S_3)$. These are the same transitions and shows $(S_1 \ ||\ S_2 )\ || \ S_3 \equiv S_1 \ ||\ (S_2 \ || \ S_3)$ for this case.\\
\indent \textbf{Subcase 5.8: }$(S_1 \ || \ S_2)\ ||\ S_3 \xrightarrow[]{\alpha} (S_1 \ || \ S_2)\ ||\ S_3$. For a $\tau$ transition, at least one system has to transition. As by \texttt{Nsys} only a receive transition can be discarded. Therefore this subcase is not possible.\\
\\
These cover all cases and therefore we have proven that associativity also holds. $\square$
\\
\textbf{3. }For the final point we want to show that a parallel composition with one of the processes being the empty process, acts the same as just the non-empty system. We will once again go by case analysis on $\alpha$. We know an empty process cannot do anything, besides discarding a receive transition as by \texttt{Nnul}. Besides the receiving transition, we cannot discard other transitions. The only possibilities are then $S_1$ performing the corresponding $\alpha$ transition.\\
\indent \textbf{Case 1: } Consider $\alpha = \Pi ! (\Tilde{v})^{ch}$.\\
\indent \indent \textbf{Subcase 1.1: } Consider \texttt{Lpar}, we get $S_1\xrightarrow[]{\Pi !(\Tilde{v})^{ch}} S_1'$. By \texttt{Sys} and our $\alpha$, we can also have that $S_1\xrightarrow[]{\Pi !(\Tilde{v})^{ch}} S_1'$ is the only possibility on the RHS.\\
\indent \indent \textbf{Subcase 1.2: } Consider \texttt{Rpar}, this would require the empty process to perform the sending action, which we know is not possible by \texttt{Nnul}.\\
\indent \textbf{Case 2: } Consider $\alpha = \Pi ? (\Tilde{v})^{ch}$.\\
\indent \indent \textbf{Subcase 2.1:} Consider \texttt{Prec}, we get $S_1\xrightarrow[]{\Pi ?(\Tilde{v})^{ch}} S_1'$ and together with \texttt{Nsys} and \texttt{Nnul} also $\Gamma : 0\xrightarrow[]{\Pi ?(\Tilde{v})^{ch}} \Gamma : 0$. By \texttt{Sys}, we also get $S_1\xrightarrow[]{\Pi ?(\Tilde{v})^{ch}} S_1'$ on the RHS.\\
\indent \indent \textbf{Subcase 2.2:} Consider \texttt{Prec}, we get $S_1\xrightarrow[]{\Pi ?(\Tilde{v})^{ch}} S_1'$ and together with \texttt{Nsys} on both this becomes $S_1\xrightarrow[]{\Pi ?(\Tilde{v})^{ch}} S_1$ and by \texttt{Nnul} also $\Gamma : 0\xrightarrow[]{\Pi ?(\Tilde{v})^{ch}} \Gamma : 0$. By \texttt{Nsys}, we also get $S_1\xrightarrow[]{\Pi ?(\Tilde{v})^{ch}} S_1$ on the RHS.\\
\indent \textbf{Case 3: } Consider $\alpha = \Pi \mathsf{G}(\Tilde{v})^{ch}$.\\
\indent \indent \textbf{Subcase 3.1: } Consider \texttt{Lget}, we get $S_1\xrightarrow[]{\Pi \mathsf{G}(\Tilde{v})^{ch}} S_1'$. By \texttt{Sys} together with our $\alpha$, we can also get $S_1\xrightarrow[]{\Pi \mathsf{S}(\Tilde{v})^{ch}} S_1'$ on the RHS.\\
\indent \indent \textbf{Subcase 3.2: } Consider \texttt{Rget}, this would require the empty process to send the get request, which we know is not possible by \texttt{Nnul}.\\
\indent \textbf{Case 4: } Consider $\alpha = \Pi \mathsf{S}(\Tilde{v})^{ch}$.\\
\indent \indent \textbf{Subcase 4.2: } Consider \texttt{Lsup}, we get $S_1\xrightarrow[]{\Pi \mathsf{S}(\Tilde{v})^{ch}} S_1'$. By \texttt{Sys}, we can also get $S_1\xrightarrow[]{\Pi \mathsf{S}(\Tilde{v})^{ch}} S_1'$ on the RHS.\\
\indent \indent \textbf{Subcase 4.2: } Consider \texttt{Rsup}, this would require the empty process to supply, which we know is not possible by \texttt{Nnul}.\\
\indent \textbf{Case 5: } Consider $\alpha = \tau$.\\
\indent \indent \textbf{Subcase 5.1: } Consider \texttt{Luni}, this would require the empty process to supply, which is not possible by \texttt{Nnul}.\\
\indent \indent \textbf{Subcase 5.2: } Consider \texttt{Runi}, this would require the empty process to send a get request, which is not possible by \texttt{Nnul}.\\
\indent \indent \textbf{Subcase 5.3: } Consider \texttt{Ltau}, we get $S_1\xrightarrow[]{\tau} S_1'$. By \texttt{Sys}, we can also get $S_1\xrightarrow[]{\Pi \mathsf{S}(\Tilde{v})^{ch}} S_1'$ on the RHS.\\
\indent \indent \textbf{Subcase 5.4: } Consider \texttt{Rtau}, this would require the empty process to perform the hidden $
\tau$ transition, which is not possible by \texttt{Nnul}.\\
As we have covered all cases we have proven that also $S_1 \ || \ \Gamma : 0 \equiv S_1$. $\square$
\\
\section{Lemma 2}
Given 2 systems $S_1$ and $S_2$ proof:\\
\begin{lemma}
Lemma 2 is regarding the non-blocking behavior of broadcasted messages.\\ 
$ch=* \land S_1 \xrightarrow[]{\pi !(\Tilde{v})^{ch}}S_1' \implies S_1\ ||\ S_2 \xrightarrow[]{\pi !(\Tilde{v})^{ch}}S_1'\ ||\ S_2'\quad $ for any $S_2$.\\
\end{lemma}

To prove this lemma we will go over all possible cases of $S_2$. The cases for this can be deducted from the grammar. Our base induction case will be $S_2=\langle \gamma, \mathsf{LS}\rangle : P$. Which means by \texttt{Lpar} we know that we need to prove $S_2 \xrightarrow[]{\pi ?(\Tilde{v})^{ch}}S_2'$. By \texttt{Nsys}, this also includes $\langle \gamma, \mathsf{LS}\rangle : P \xmapsto[]{\widetilde{\pi ?(\Tilde{v})^{ch}}}\langle \gamma, \mathsf{LS}\rangle : P$. Because it is about non-blocking behavior, we can assume that it is possible for $S_2'=S_2$.\\
\textbf{Case 1: } $P=0$. An empty process can only discard a message as per \texttt{Nnul}, by \texttt{Nsys} we then get $\Gamma : 0 \xmapsto[]{\pi ?(\Tilde{v})^{ch}} \Gamma : 0$. As \texttt{Nnul} has no premises, we have $S_1\ ||\ S_2 \xrightarrow[]{\pi !(\Tilde{v})^{ch}}S_1'\ ||\ S_2'$, making it non-blocking as required.\\
\textbf{Case 2: } $P=\Pi ! (\Tilde{E})^{E'}.U $. We can only apply \texttt{Nsnd}, as we know we need $S_2 \xrightarrow[]{\pi ?(\Tilde{v})^{ch}}S_2'$. We can simply apply \texttt{Nsnd} in this case, as it has no premises. We have $S_1\ ||\ S_2 \xrightarrow[]{\pi !(\Tilde{v})^{ch}}S_1'\ ||\ S_2'$, making it non-blocking as required. \\
\textbf{Case 3: } $P=\Pi ? (\Tilde{x})^{E'}.U $.\\
\indent \textbf{Subcase 3.1: } Consider \texttt{Rec} if $\llbracket E ' \rrbracket_\gamma = * \land gamma \models \{ \Pi [\Tilde{v}/\Tilde{x}] \} \land \gamma \models \pi ' $, $S_2$ will receive the message, resulting in $S_2 \xrightarrow[]{\pi ?(\Tilde{v})^{ch}}S_2'$. \\
\indent \textbf{Subcase 3.2: } Consider \texttt{Nrec}, we know that $ch\in \mathsf{LS}$, as the broadcast channel always is. This means if \\
\indent $\bigl(ch = * \ \land \ (\gamma \not\models \{ \Pi [\Tilde{v}/\Tilde{x}]\} \ \lor \ \gamma \not\models \pi ')\bigr)$, we can apply \texttt{Nrec} with \texttt{Nsys} resulting in $S_2 \xrightarrow[]{\pi ?(\Tilde{v})^{ch}}S_2$.\\
As we have that $ch=*$ and we know the broadcast channel is always in $\mathsf{LS}$, we get that either $\gamma \models \{ \Pi [\Tilde{v}/\Tilde{x}] \} \land \gamma \models \pi ' $ or $\gamma \not\models \{ \Pi [\Tilde{v}/\Tilde{x}]\} \ \lor \ \gamma \not\models \pi '$ has to hold. This is a tautology and thus we then have $S_1\ ||\ S_2 \xrightarrow[]{\pi !(\Tilde{v})^{ch}}S_1'\ ||\ S_2'$, making it non-blocking as required.\\
\textbf{Case 4: } $P=\Pi \mathsf{G} (\Tilde{x})^{E'}.U $. Consider \texttt{Nget}, which is the only applicable rule for this $P$, as we know we need $S_2 \xrightarrow[]{\pi ?(\Tilde{v})^{ch}}S_2'$. There are no premises, meaning we can apply this rule regardless. We then have $S_1\ ||\ S_2 \xrightarrow[]{\pi !(\Tilde{v})^{ch}}S_1'\ ||\ S_2'$, making it non-blocking as required.\\
\textbf{Case 5: } $P=\mathsf{S} (\Tilde{E})^{E'}.U $. Consider \texttt{Nsup}, which is the only applicable rule for this $P$, as we know we need $S_2 \xrightarrow[]{\pi ?(\Tilde{v})^{ch}}S_2'$. The premise holds as we know $ch=*$. We then have $S_1\ ||\ S_2 \xrightarrow[]{\pi !(\Tilde{v})^{ch}}S_1'\ ||\ S_2'$, making it non-blocking as required.\\
\textbf{Case 6: } $P=\langle \Pi \rangle P $.\\
\indent \textbf{Subcase 6.1:} Consider \texttt{Grd}, if $\gamma \models \Pi$ and $\langle \gamma, \mathsf{LS}\rangle : P \xmapsto[]{\alpha}\langle \gamma, \mathsf{LS}\rangle : P$, we can receive the message. As we know we need $S_2 \xrightarrow[]{\pi ?(\Tilde{v})^{ch}}S_2'$, we know $\alpha$, has to be the receiving transition, which we covered in case 3.\\
\indent \textbf{Subcase 6.2:} Consider \texttt{Blk}, in case $\gamma \not\models \pi '$, we can discard. \\
\indent \textbf{Subcase 6.3:} Consider \texttt{Stc}, if $\gamma \models \Pi$ and we can do $\langle \gamma , \mathsf{LS} \rangle : P \xmapsto[]{\widetilde{\pi ' ? (\Tilde{v})^{ch}}}\langle \gamma , \mathsf{LS} \rangle : P $, which is covered in the previous \indent cases, we can discard. \\
We then have $\bigl( \gamma \models \Pi \land (\langle \gamma , \mathsf{LS} \rangle : P \xmapsto[]{\widetilde{\pi ' ? (\Tilde{v})^{ch}}}\langle \gamma , \mathsf{LS} \rangle : P \lor \langle \gamma , \mathsf{LS} \rangle : P \xmapsto[]{\pi ' ? (\Tilde{v})^{ch}}\langle \gamma , \mathsf{LS} \rangle : P )\bigr) \lor \gamma \not\models \Pi$, which is a tautology. We then have $S_1\ ||\ S_2 \xrightarrow[]{\pi !(\Tilde{v})^{ch}}S_1'\ ||\ S_2'$, making it non-blocking as required.\\
\textbf{Case 7: } $P=P_1+P_2 $. \\
\indent \textbf{Subcase 7.1: } Consider \texttt{Lor}, if $\langle \gamma , \mathsf{LS} \rangle : P_1 \xmapsto[]{\pi ' ? (\Tilde{v})^{ch}}\langle \gamma , \mathsf{LS} \rangle : P_1'$, we have $\langle \gamma , \mathsf{LS} \rangle : P_1+P_2 \xmapsto[]{\pi ' ? (\Tilde{v})^{ch}}\langle \gamma , \mathsf{LS} \rangle : P_1'$. Note we use $\alpha = \pi ' ? (\Tilde{v})^{ch}$ because we know we need $S_2 \xrightarrow[]{\pi ?(\Tilde{v})^{ch}}S_2$.\\
\indent \textbf{Subcase 7.2: } Consider \texttt{Ror}, if $\langle \gamma , \mathsf{LS} \rangle : P_2 \xmapsto[]{\pi ' ? (\Tilde{v})^{ch}}\langle \gamma , \mathsf{LS} \rangle : P_2'$, we have $\langle \gamma , \mathsf{LS} \rangle : P_1+P_2 \xmapsto[]{\pi ' ? (\Tilde{v})^{ch}}\langle \gamma , \mathsf{LS} \rangle : P_2'$. Note we use $\alpha = \pi ' ? (\Tilde{v})^{ch}$ because we know we need $S_2 \xrightarrow[]{\pi ?(\Tilde{v})^{ch}}S_2$.\\
\indent \textbf{Subcase 7.3: } Consider \texttt{Nor}, by the previous cases we have shown $\langle \gamma, \mathsf{LS}\rangle : P \xmapsto[]{\widetilde{\pi ?(\Tilde{v})^{ch}}}\langle \gamma, \mathsf{LS}\rangle : P$ is non-blocking. This means for $\langle \gamma, \mathsf{LS}\rangle : P_1 \xmapsto[]{\widetilde{\pi ?(\Tilde{v})^{ch}}}\langle \gamma, \mathsf{LS}\rangle : P_1$ and $\langle \gamma, \mathsf{LS}\rangle : P_2 \xmapsto[]{\widetilde{\pi ?(\Tilde{v})^{ch}}}\langle \gamma, \mathsf{LS}\rangle : P_2$ and thus for \\
\indent $\langle \gamma , \mathsf{LS} \rangle : P_1+P_2 \xmapsto[]{\pi ' ? (\widetilde{\Tilde{v})^{ch}}}\langle \gamma , \mathsf{LS} \rangle : P_1+P_2$ it is also non-blocking for broadcast.\\
As these are all cases for $P=P_1+P_2$, we then have $S_1\ ||\ S_2 \xrightarrow[]{\pi !(\Tilde{v})^{ch}}S_1'\ ||\ S_2'$, making it non-blocking as required.
\textbf{Case 8: } $P=K(\Tilde{x}) $.\\ 
\indent \textbf{Subcase 8.1: } Consider \texttt{Def}, if $\langle \gamma , \mathsf{LS} \rangle : P \xmapsto[]{\pi ' ? (\Tilde{v})^{ch}}\langle \gamma , \mathsf{LS} \rangle : P'$, we can apply this rule and receive the message.\\
\indent \textbf{Subcase 8.2: } Consider \texttt{Ndef}, if $\langle \gamma , \mathsf{LS} \rangle : P \xmapsto[]{\widetilde{\pi ' ? (\Tilde{v})^{ch}}}\langle \gamma , \mathsf{LS} \rangle : P$, which by our prevous cases has been shown to be non-blocking for broadcast, we can discard it.\\
Both cases result in $S_1\ ||\ S_2 \xrightarrow[]{\pi !(\Tilde{v})^{ch}}S_1'\ ||\ S_2'$, making it non-blocking as required.
\\
For the inductive step, we take $S_2=S_i || S_j $. This inductive step follows from \texttt{Prec}. Which can be applied in combination with \texttt{Lpar} and together with the base cases showing it is non-blocking for broadcasted messages.$\square$\\
\section{Lemma 3}
\begin{lemma}
    Lemma 3 shows the blocking of multi cast channels.\\
$ch\not = * \land S_1 \xrightarrow[]{\pi !(\Tilde{v})^{ch}}S_1' \implies S_1\ ||\ S_2 \xrightarrow[]{\pi !(\Tilde{v})^{ch}}S_1'\ ||\ S_2'\quad$ for any $S_2$ such that:
\begin{enumerate}
    \item $ch \in \mathsf{LS} \land S_2 \xrightarrow[]{\pi ?(\Tilde{v})^{ch}}S_2'$\\
    $\lor$
    \item $ch \not \in \mathsf{LS}$
\end{enumerate}
\end{lemma}
We will use the same cases as in \ref{lem2} but divide them into the two subcases. Once again to prove this lemma we will go over all possible cases of $S_2$. The cases for this can be deducted from the grammar. Our base induction case will be $S_2=\langle \gamma, \mathsf{LS}\rangle : P$. Which means by \texttt{Lpar} we know that we need to prove $S_2 \xrightarrow[]{\pi ?(\Tilde{v})^{ch}}S_2'$. By \texttt{Nsys}, this also includes $\langle \gamma, \mathsf{LS}\rangle : P \xmapsto[]{\widetilde{\pi ?(\Tilde{v})^{ch}}}\langle \gamma, \mathsf{LS}\rangle : P$. Because it is about blocking behavior, we can assume that it is possible for $S_2'=S_2$ if appropriate.\\
\textbf{Case 1: } $P=0$. Consider \texttt{Nnul}, this rule has not premises. We have that an empty process cannot block an other process. An empty process cannot progress to anything and therefore, we do not want it blocking a multi cast channel. For both subcases this is the same and we have $S_1\ ||\ S_2 \xrightarrow[]{\pi !(\Tilde{v})^{ch}}S_1'\ ||\ S_2'$ as required.\\
\textbf{Case 2: } $P=\Pi ! (\Tilde{E})^{E'}.U$. Consider \texttt{Nsnd}, as we need to get $S_2 \xrightarrow[]{\pi ?(\Tilde{v})^{ch}} S_2'$. Just like the last case, the reasoning for both subcases is the same as we have no premises we can always apply this rule. This is desirable because we can only send one message at a time and do not want it to block other messages being send. We then have we have $S_1\ ||\ S_2 \xrightarrow[]{\pi !(\Tilde{v})^{ch}}S_1'\ ||\ S_2'$ as required.\\
\textbf{Case 3: } $P=\Pi ? (\Tilde{E})^{E'}.U$.\\
\indent \textbf{Subcase 3.1:}\\
\indent \indent \textbf{Subcase 3.1.1: } Consider \texttt{Rec}, we have that if $\llbracket E ' \rrbracket_\gamma = ch \land \gamma \models \{ \Pi [\Tilde{v}/\Tilde{x}] \} \land \gamma \models \pi ' \land ch \in \mathsf{LS} $ then $\langle \gamma , \mathsf{LS} \rangle : \Pi ? (\Tilde{x})^{E'}.U\xmapsto[]{\pi '? (\Tilde{v})^{ch}} \llbrace \langle \gamma , \mathsf{LS}\rangle : [\Tilde{v}/\Tilde{x}]U\rrbrace$. We know $ch\in \mathsf{LS}$, so we are left with the premises $\llbracket E ' \rrbracket_\gamma = ch \land \gamma \models \{ \Pi [\Tilde{v}/\Tilde{x}] \} \land \gamma \models \pi '$.\\ 
\indent \indent \textbf{Subcase 3.1.2: } Consider \texttt{Nrec}, as we know $ch\in \mathsf{LS}$, leaving us with $\bigl(ch = * \ \land \ (\gamma \not\models \{ \Pi [\Tilde{v}/\Tilde{x}]\} \ \lor \ \gamma \not\models \pi ')\bigr)$. We know $ch\not = *$, meaning $\langle \gamma , \mathsf{LS} \rangle : \Pi ? (\Tilde{x})^{E'}.U\xmapsto[]{\widetilde{\pi '? (\Tilde{v})^{ch}}} \langle \gamma , \mathsf{LS} \rangle : \Pi ? (\Tilde{x})^{E'}.U$ is never possible in this case.\\
What we are left with it that for it to be blocking either $\llbracket E ' \rrbracket_\gamma \not = ch$, meaning it is expecting a message from a different channel, which means the process it is doing multiple jobs which is undesirable and thus appropriate blocking. Or $\gamma \not \models \{ \Pi [\Tilde{v}/\Tilde{x}] \} \land \gamma \not \models \pi '$, which means a process is not ready yet to receive and that the system should wait or something went wrong, which is why it is appropriate to be blocking.\\
\indent \textbf{Subcase 3.2} \\
\indent \indent \textbf{Subcase 3.2.1: } Consider \texttt{Rec}, we have that $ch \not \in \mathsf{LS}$, meaning this is not possible as $ch \in \mathsf{LS}$ is a premise. \\
\indent \indent \textbf{Subcase 3.1.2: } Consider \texttt{Nrec}, our premise is $ch \not\in \mathsf{LS} \lor \bigl(ch = * \ \land \ (\gamma \not\models \{ \Pi [\Tilde{v}/\Tilde{x}]\} \ \lor \ \gamma \not\models \pi ')\bigr)$. We have that $ch\not \in \mathsf{LS}$, thus we can always perform this transition, making it non-blocking if the process is not listening to the channel, which is desirable as they should not intervene with processes they are not part of.\\
This shows that in this case it is blocking the multi cast channels appropriately.\\
\textbf{Case 4: } $P=\Pi \mathsf{G} (\Tilde{x})^{E'}.U $. Consider \texttt{Nget}, as we need to get $S_2 \xrightarrow[]{\pi ?(\Tilde{v})^{ch}} S_2'$. Just like the first two cases, the reasoning for both subcases is the same as we have no premises we can always apply this rule. This is desirable because we can only send one message at a time and do not want it to block other messages being send. We then have we have $S_1\ ||\ S_2 \xrightarrow[]{\pi !(\Tilde{v})^{ch}}S_1'\ ||\ S_2'$ as required.\\
\textbf{Case 5: } $P=\mathsf{S} (\Tilde{x})^{E'}.U$.
\indent \textbf{Subcase 5.1: } Consider \texttt{Nsup}, as we know we need $S_2 \xrightarrow[]{\pi ?(\Tilde{v})^{ch}} S_2'$. We know $ch\not = *$, leaving us with the premise $ch \not \in \mathsf{LS}$, which we know is false as well. This means it would block the multi-cast message, which is desirable as unicast and multicast should be done on seperate channels.\\
\indent \textbf{Subcase 5.2: } Consider \texttt{Nsup}, as we know we need $S_2 \xrightarrow[]{\pi ?(\Tilde{v})^{ch}} S_2'$. We know $ch\not = *$, leaving us with the premise $ch \not \in \mathsf{LS}$, which we know is true. This means it does not block the message because when it is not listening to the channel, which is desirable as they should not intervene with processes they are not part of. We then have we have $S_1\ ||\ S_2 \xrightarrow[]{\pi !(\Tilde{v})^{ch}}S_1'\ ||\ S_2'$ as required.\\
\textbf{Case 6: } $P=\langle \Pi \rangle P$. None of the cases depend on $ch$ directly, meaning we can cover both cases at the same time.\\
\indent \textbf{Subcase 6.1: } Consider \texttt{Grd}, we have covered the $\gamma \models \Pi$ and $\langle \gamma, \mathsf{LS}\rangle : P \xmapsto[]{\pi '? (\Tilde{v})^{ch}}\langle \gamma, \mathsf{LS}\rangle : P$ premise in the previous cases, meaning if $\gamma \models \Pi$ holds, it will only block when appropriate. This is the case for both of our subcases\\
\indent \textbf{Subcase 6.1: } Consider \texttt{Stc}, if $\gamma \not \models \Pi$, it will discard regardless. This is desirable because an internal guard means a process should not be participating (yet), meaning it should not block the other processes.\\
\indent \textbf{Subcase 6.1: } Consider \texttt{Stc}, we have covered the $\gamma \models \Pi$ and $\langle \gamma, \mathsf{LS}\rangle : P \xmapsto[]{\widetilde{\pi '? (\Tilde{v})^{ch}}}\langle \gamma, \mathsf{LS}\rangle : P$ premise in the previous cases, meaning if $\gamma \models \Pi$ holds, it will only discard when appropriate.\\
We have covered all of the subcases and we have $S_1\ ||\ S_2 \xrightarrow[]{\pi !(\Tilde{v})^{ch}}S_1'\ ||\ S_2'$ as required, only blocking when appropriate as covered by the previous cases. \\
\textbf{Case 7: } $P=P_1 + P_2$. Just like the previous case none of the cases depend on $ch$ directly, meaning we can cover both cases at the same time.\\
\indent \textbf{Subcase 7.1: } Consider \texttt{Lor}, if the premise $\langle \gamma , \mathsf{LS} \rangle : P_1 \xmapsto[]{\pi ' ? (\Tilde{v})^{ch}}\langle \gamma , \mathsf{LS} \rangle : P_1'$ holds, we know it accepts and we get $\langle \gamma , \mathsf{LS} \rangle : P_1+P_2 \xmapsto[]{\pi ' ? (\Tilde{v})^{ch}}\langle \gamma , \mathsf{LS} \rangle : P_2'$. \\
\indent \textbf{Subcase 7.2: } Consider \texttt{Ror}, if the premise $\langle \gamma , \mathsf{LS} \rangle : P_2 \xmapsto[]{\pi ' ? (\Tilde{v})^{ch}}\langle \gamma , \mathsf{LS} \rangle : P_2'$ holds, we know it accepts and we get $\langle \gamma , \mathsf{LS} \rangle : P_1+P_2 \xmapsto[]{\pi ' ? (\Tilde{v})^{ch}}\langle \gamma , \mathsf{LS} \rangle : P_1'$.\\
\indent \textbf{Subcase 7.3: } Consider \texttt{Nor}, we know $\langle \gamma, \mathsf{LS}\rangle : P_1 \xmapsto[]{\widetilde{\pi ?(\Tilde{v})^{ch}}}\langle \gamma, \mathsf{LS}\rangle : P_1$ and $\langle \gamma, \mathsf{LS}\rangle : P_2 \xmapsto[]{\widetilde{\pi ?(\Tilde{v})^{ch}}}\langle \gamma, \mathsf{LS}\rangle : P_2$ is only possible when it should be non-blocking. \\
As these are all cases for $P=P_1+P_2$, we then have $S_1\ ||\ S_2 \xrightarrow[]{\pi !(\Tilde{v})^{ch}}S_1'\ ||\ S_2'$, making it blocking only if appropriate as required.\\
\textbf{Case 8: } $P=K(\Tilde{x})$. Just like the previous cases none of the cases depend on $ch$ directly, meaning we can cover both cases at the same time.\\
\indent \textbf{Subcase 8.1: } Consider \texttt{Def}, if $\langle \gamma , \mathsf{LS} \rangle : P \xmapsto[]{\pi ' ? (\Tilde{v})^{ch}}\langle \gamma , \mathsf{LS} \rangle : P'$, we can apply this rule and receive the message.\\
\indent \textbf{Subcase 8.2: } Consider \texttt{Ndef}, if $\langle \gamma , \mathsf{LS} \rangle : P \xmapsto[]{\widetilde{\pi ' ? (\Tilde{v})^{ch}}}\langle \gamma , \mathsf{LS} \rangle : P'$, we can discard the message, which by the previous cases can only be done when appropriate.\\
As these are all cases for $P=K(\Tilde{x})$, we then have $S_1\ ||\ S_2 \xrightarrow[]{\pi !(\Tilde{v})^{ch}}S_1'\ ||\ S_2'$, making it blocking only if appropriate as required.\\
\\
For the inductive step, we take $S_2=S_i\ ||\ S_j $. This inductive step follows from \texttt{Prec}. Which can be applied in combination with \texttt{Lpar} and together with the base cases shows it is blocking for multi-cast messages when it should be.$\square$\\
